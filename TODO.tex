
% TODO: reformat chapter style with package titlesec
%\titleformat{\section}
%{\filcenter\normalfont\Large\bfseries}
%{\chaptertitlename~\thechapter} {0.5em} {}

% TODO: Bei erstem Vorkommen von Golang, REferenz auf deren Website setzen.

% TODO: Als abkürzungen festlegen und verlinken: HP, IBM, LXC
% TODO: Ins Glossar: chroot, Jails?, Zones?, System Calls (oder eigenen kapitel?), Apache 2.0 Lizenz, Golang, COW, High Performance Computing, embedded systems
% TODO: Klaeren: MAC abkuerzung doppeltdeutig: Media/Mandatory Access Control
%       --> wie wird das im Abkürzungsverzeichnis gemanaged?

% TODO: Klären: Codeblocks (latex lstlisting) auch in ein "Codeblockverzeichnis" ?
%       Bzw. iwo zentral auflisten? Bisher sinds 2 (Dockerfile + cgroups kapitel)

% TODO: ADD im Kapitel Dockerfile hinzufügen, da im Kapitel "Images" erwähnt.

% TODO: Isolierung-Unterkapitel in eine sinnvolle Reihenfolge bringen und 6 raussuchen (sind 7 aktuell)

% TODO: Auf Schreibweise einigen: containerbasiert oder Container-basiert

% TODO: Klaeren ob Sachen wie "IBM" und "HP" und "LXC" und "OpenVZ" als Abkürzungen gelistet werden sollen mit ausführlichem Namen
% TODO: Klaeren ob Referenzen zu z.B. Golang, LXC bei erstem vorkommen gestzt werden oder bei JEDEM vorkommen des namen?

% TODO: Rework abstract syntax. Its shitty

% TODO: Unterschiedung von Produkt Docker und Unternehmen Docker. Das Unternehmen evtl. kursiv schreiben oder mit (TM)-Zeichen am Ende
          % --> Klaeren ob das okay so ist, auch obs ok ist das in "struktur der arbeit" zu schreiben

% \linespread{1.3}            % 1.5x line spacing
% \linespread{1.6}            % Double line spacing
% \hfill test                 % insert horizontal stretched space
% \vfill test                 % insert vertical stretched space
% ,,German quotation marks``  % deutsche Anfuehrungszeichen

%\subsection{Subsektion blabla}
%\subsubsection{Subsubsektion blablabla}
%\paragraph{Paragraph soundso}
%\subparagraph{SubParagraph soundso}
