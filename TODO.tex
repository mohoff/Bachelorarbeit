
% TODO: reformat chapter style with package titlesec
%\titleformat{\section}
%{\filcenter\normalfont\Large\bfseries}
%{\chaptertitlename~\thechapter} {0.5em} {}

% TODO: Bei erstem Vorkommen von Golang, REferenz auf deren Website setzen.

% TODO: Als abkürzungen festlegen und verlinken: HP, IBM, LXC
% TODO: Ins Glossar: chroot, Jails?, Zones?, System Calls (oder eigenen kapitel?), Apache 2.0 Lizenz, Golang, COW, High Performance Computing, embedded systems, Socket (s.1149,s.1162f.,s.1176,s.1235{linuxInterface}), Bash
% TODO: Klaeren: MAC abkuerzung doppeltdeutig: Media/Mandatory Access Control
%       --> wie wird das im Abkürzungsverzeichnis gemanaged?

% TODO: Klären: Codeblocks (latex lstlisting) auch in ein "Codeblockverzeichnis" ?
%       Bzw. iwo zentral auflisten? Bisher sinds 2 (Dockerfile + cgroups kapitel)

% TODO: ADD im Kapitel Dockerfile hinzufügen, da im Kapitel "Images" erwähnt.

% TODO: Isolierung-Unterkapitel in eine sinnvolle Reihenfolge bringen und 6 raussuchen (sind 7 aktuell)

% TODO: Auf Schreibweise einigen: containerbasiert oder Container-basiert ... noch viele andere Woerter wo konsistent geschrieben werden muessen!

% TODO: Dockers AppArmor- und Seccomp-Profile in Appendix?
%       --> Oder intext? Seccomp ist ellenlang, da viele Systemcall drin sind. AppArmor geht inline.
%       --> SELinux profile?

% TODO: Eigenes Grundlagenkapitel? :
%         /proc Filesystem (evtl Auflistung der Files im Appendix) .... auch in \cite[S.42,S.223-227,S.231]{linuxInterface} und
%         DAC (normale Linux-Userauthentifizierung) .... auch in \cite(grundlagenkapitel){linuxInterface}
%             --> Users and Groups in \cite[S.153]{linuxInterface}, \cite[S.183]{linuxInterface}
%         System Calls (evtl Auflistung im Appendix) .... auch in \cite[S.43ff.,S.68]{linuxInterface}
%     PID (Process ID) ..... auch in \cite[S.114f.]{linuxInterface}

% TODO: Local DoS: Fork-bomb (siehe \cite[S.793]{linuxInterface})

% TODO: Klaeren ob Sachen wie "IBM" und "HP" und "LXC" und "OpenVZ" als Abkürzungen gelistet werden sollen mit ausführlichem Namen
% TODO: Klaeren ob Referenzen zu z.B. Golang, LXC bei erstem vorkommen gestzt werden oder bei JEDEM vorkommen des namen?

% TODO: Rework abstract syntax. Its shitty

% TODO: Unterschiedung von Produkt Docker und Unternehmen Docker. Das Unternehmen evtl. kursiv schreiben oder mit (TM)-Zeichen am Ende
          % --> Klaeren ob das okay so ist, auch obs ok ist das in "struktur der arbeit" zu schreiben

% TODO: Eigenes groses Kapitel "Docker im Vergleich zu anderen Containertechnologien"
%       Hauptsaechlich Docker und CoreOS' rkt vergleichen (rkt bietet KVM support an: man hat die wahl zwischen hardwarevirtualisierung und containern)

% TODO: im github repo von runc, dem neuen containerstandard werden technologien (namesapces,cgroups,selinux,apparmor) abstrahiert. es ist von "zusätzlichen isolatoren" die rede (siehe SPEC.md oder so in runC), damit ist selinux,seccomp,apparmor,caps gemeint
%       Außerdem mehr die Rede von "pods", nicht "containern". Ein pod hat ein oder mherere container enthalten. --> Shift von Container zu Pods. Docker wird unwichtiger, Kubernetes wird wichtiger.

% \linespread{1.3}            % 1.5x line spacing
% \linespread{1.6}            % Double line spacing
% \hfill test                 % insert horizontal stretched space
% \vfill test                 % insert vertical stretched space
% ,,German quotation marks``  % deutsche Anfuehrungszeichen

%\subsection{Subsektion blabla}
%\subsubsection{Subsubsektion blablabla}
%\paragraph{Paragraph soundso}
%\subparagraph{SubParagraph soundso}
