
%%%%% LAST STEPS %%%%%

% TODO: texttt{}-Bloecke sauber zeilenbrechen. --> mit usepackage oder anderem workaround?
% --> Zeilenumbruch davor (Aufgabe am Ende kurz vor Abgabe)

% TODO: Acronym "MAC" kommt 3 mal vor... Sicherstellen, dass alle 3 mal richtig referenziert ist

% TODO: Hyphenations am Ende machen, wenn Inhalt steht.

% TODO: Sicherstellen, dass alle Glossar-eintraege auch im text verwendet werden. Wenn nicht, entsprechend loeschen.


%%%%% STRUKTUR und STIL %%%%%

% TODO: einheitliche Schreibweise von control groups, capabilities, namespaces
% --> normal schreiben, kursiv mit \emph{..}, oder monotype mit \texttt{..} ???

% TODO: Klaeren, ob Bilder mit englischen Texten selbst nachgebaut werden muessen mit eingedeutschten Begriffen?
% --> Laut Patrick englisch lassen

% TODO: Kommandozeilen-Screenshots besser als Codeblocks/lstlistings? Oder ist ok mit violettem Hintergrund...(bei eigenen Screenshots)

% TODO: Sicherheitsziel der Aktualität als Eigenform der Integrität? So in Grundlagenkapitel "Sicherheitsziele" schreiben, oder aus Aktualität sogar eigenes Sicherheitsziel machen?

%%%%% LATEX %%%%%







% (TODO): Rework abstract syntax. Its shitty. Gibts da was gutes als Vorlage?



%%%%% richtige TODOs %%%%%

% (TODO): Begriff "Microservices" neben CaaS auch noch als Buzzword erwähnen

% TODO: Linux security/lsm namespace: https://lwn.net/Articles/623575/

% TODO: Socket (s.1149,s.1162f.,s.1176,s.1235{linuxInterface}),  SaaS, PaaS, IaaS, (CaaS), CI, CD

% TODO: reformat chapter style with package titlesec
%\titleformat{\section}
%{\filcenter\normalfont\Large\bfseries}
%{\chaptertitlename~\thechapter} {0.5em} {}

% (TODO): ADD im Kapitel Dockerfile hinzufügen, da im Kapitel "Images" erwähnt.

% (TODO): Quelle \cite{nist} weiter untersuchen. V.a. S.15,23 hat gute Infos





% TODO: fuer jeden github-repo link einen author finden. commited lines finden und hauptcontributor mit author="" angeben.





%%%% ERLEDIGTE TODOS %%%%%

% DO NOT: Eigenes groses Kapitel "Docker im Vergleich zu anderen Containertechnologien"
%       Hauptsaechlich Docker und CoreOS' rkt vergleichen (rkt bietet KVM support an: man hat die wahl zwischen hardwarevirtualisierung und containern)

% Unterschiedung von Produkt Docker und Unternehmen Docker. Das Unternehmen evtl. kursiv schreiben oder mit (TM)-Zeichen am Ende
          % --> Klaeren ob das okay so ist, auch obs ok ist das in "struktur der arbeit" zu schreiben

% Isolierung-Unterkapitel in eine sinnvolle Reihenfolge bringen

% DO NOT: Bei erstem Vorkommen von Golang, REferenz auf deren Website setzen.

% DO NOT: Als abkürzungen festlegen und verlinken: HP, IBM

% Nirgendwo Third-Party schreiben, da guter dt. Begriff mit Drittanbieter/Dritte existiert.

% Begriff Host von dem in der Netzwerktechnik im OSI-Modell abgrenzen

% Konsistente Schreibeweise von "RedHat"/"Red Hat"

% Begriff "System" definieren. Gesamtheit von Hostsystem und Gastsystem. Meint Betriebssystem inklusive daraus gestartete Container

% DONE: grsecurity und SMACK und TOMOYO auch mit rein zum erklaeren oder besser begruenden warum sie alle keine direkte relevanz zu docker haben (keine profile, sondern einfach kernel patches, die ohne konfiguration auskommen)

% Tabelle in Einleitungskapitel "Virtualisierung" fixen. Da muss Seitenumbruch rein iwie.
% --> Entweder richtiges Package finden, sonst zwei Tabellen auf 2 Seiten (Aufgabe am Ende kurz vor Abgabe)
% Mit tex-files von Karlo vergleichen, der das angeblich mit longtables hinbekommen hat...

% Klaeren: MAC abkuerzung doppeltdeutig: Media/Mandatory Access Control
% --> wie wird das im Abkürzungsverzeichnis gemanaged?

% Eigenes Grundlagenkapitel? :
%         /proc Filesystem (evtl Auflistung der Files im Appendix) .... auch in \cite[S.42,S.223-227,S.231]{linuxInterface} und
%         DAC (normale Linux-Userauthentifizierung) .... auch in \cite(grundlagenkapitel){linuxInterface}
%             --> Users and Groups in \cite[S.153]{linuxInterface}, \cite[S.183]{linuxInterface}
%         System Calls (evtl Auflistung im Appendix) .... auch in \cite[S.43ff.,S.68]{linuxInterface}
%     PID (Process ID) ..... auch in \cite[S.114f.]{linuxInterface}

% Aktuell gibts ein Glossar und ein Abkürzungsverzeichnis. Glossar für "erweitertes Grundlagenkapitel", in dem Begriffe in 1-3 Sätzen erklärt werden
% --> Passt

% Kapitel "Struktur der Arbeit" und "Forschungsfrage" überschneiden sich, ist das Okay? In beiden Kapiteln wird z.b. abgegrenzt
% --> Kleines Kapitel 1.2 mit "Ziel der Arbeit" in Kurzfassung. Auf detaillierte Version nach Grundlagenkapitel hinweisen

% Klären: Codeblocks (latex lstlisting) auch in ein "Codeblockverzeichnis" ?
% --> Brauchts nicht.

% Auf Schreibweise einigen:
%           containerbasiert oder Container-basiert
%           hypervisorbasiert oder Hypervisor-basiert
% Synonyme erlaubt? Gast, Gastinstanz, Gastsystem. Ist es okay, wenn die beim ersten Vorkommen als Synonyme gekennzeichnet sind?
% ... noch viele andere Woerter, die konsistent geschrieben werden muessen!
% --> Wichtig ist Konsistenz. Außerdem Gastsystem besser wie Gast. Gast zu umgangssprachlich.

% Dockers SELinux-, AppArmor- und Seccomp-Profile in Appendix?
% --> Reihenfolge ändern. Erst AppArmor mit Docker-Profil. Dann SELinux und Seccomp ohne Profil intext, sondern mit bloßer Verlinkenung. Nur begründen, warum die Profile dazu nicht im Text sind (zu lang)

% Klaeren ob Sachen wie "IBM" und "HP" und "LXC" und "OpenVZ" als Abkürzungen gelistet werden sollen mit ausführlichem Namen
% --> Firmennamen nicht, Produktnamen wie LXC und OpenVZ schon.

% Klaeren ob Referenzen zu z.B. Golang, LXC bei erstem vorkommen gestzt werden oder bei JEDEM vorkommen des namen?
% --> Mindestens beim ersten Vorkommen. Sonst auch bei jedem Vorkommen moeglich. (Aufgabe kurz vor Abgabe)

% Kapitel "Fragestellung/Forschungsfrage" geht erst S.23 los... Problem?
% --> Evtl Docker-Einleitung etw. kuerzen am Ende (Aufgabe kurz vor Ende)

% Klaeren: MAC abkuerzung doppeltdeutig: Media/Mandatory Access Control
% --> wie wird das im Abkürzungsverzeichnis gemanaged?




% \linespread{1.3}            % 1.5x line spacing
% \linespread{1.6}            % Double line spacing
% \hfill test                 % insert horizontal stretched space
% \vfill test                 % insert vertical stretched space
% ,,German quotation marks``  % deutsche Anfuehrungszeichen

%\subsection{Subsektion blabla}
%\subsubsection{Subsubsektion blablabla}
%\paragraph{Paragraph soundso}
%\subparagraph{SubParagraph soundso}
