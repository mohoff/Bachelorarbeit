
%%%%% STRUKTUR und STIL %%%%%

% TODO: Aktuell gibts ein Glossar und ein Abkürzungsverzeichnis. Glossar für "erweitertes Grundlagenkapitel", in dem Begriffe in 1-3 Sätzen erklärt werden

% TODO: Kapitel "Struktur der Arbeit" und "Forschungsfrage" überschneiden sich, ist das Okay? In beiden Kapiteln wird z.b. abgegrenzt

% TODO: Klären: Codeblocks (latex lstlisting) auch in ein "Codeblockverzeichnis" ?
%       Bzw. iwo zentral auflisten? Bisher sinds 2 (Dockerfile + cgroups kapitel)

% TODO: Auf Schreibweise einigen:
%           containerbasiert oder Container-basiert
%           hypervisorbasiert oder Hypervisor-basiert
% Synonyme erlaubt? Gast, Gastinstanz, Gastsystem. Ist es okay, wenn die beim ersten Vorkommen als Synonyme gekennzeichnet sind?
% ... noch viele andere Woerter, die konsistent geschrieben werden muessen!

% TODO: Dockers SELinux-, AppArmor- und Seccomp-Profile in Appendix?
%       --> Oder intext? Seccomp ist ellenlang, da viele Systemcall drin sind. AppArmor geht inline.
%       --> SELinux profile? SELinux auch zu lang mit 2-3 Files unter 1000Zeilen+.
%       --> Eigtl nur AppArmor vertretbar, da das auf 1-2 Seiten passt. Dann ists aber nicht konsistent zu SELinux und SEccomp...

% TODO: Klaeren ob Sachen wie "IBM" und "HP" und "LXC" und "OpenVZ" als Abkürzungen gelistet werden sollen mit ausführlichem Namen
% TODO: Klaeren ob Referenzen zu z.B. Golang, LXC bei erstem vorkommen gestzt werden oder bei JEDEM vorkommen des namen?

% TODO: Kapitel "Fragestellung/Forschungsfrage" geht erst S.23 los... Problem?



%%%%% LATEX %%%%%

% TODO: Klaeren: MAC abkuerzung doppeltdeutig: Media/Mandatory Access Control
%       --> wie wird das im Abkürzungsverzeichnis gemanaged?

% TODO: texttt{}-Bloecke sauber zeilenbrechen. --> mit usepackage oder anderem workaround?

% TODO: Tabelle in Einleitungskapitel "Virtualisierung" fixen. Da muss Seitenumbruch rein iwie.

% TODO: Rework abstract syntax. Its shitty. Gibts da was gutes als Vorlage?


%%%%% richtige TODOs %%%%%

% TODO: Ins Glossar: chroot, Jails?, Zones?, System Calls (oder eigenen kapitel?), Apache 2.0 Lizenz, Golang, COW, High Performance Computing, embedded systems, Socket (s.1149,s.1162f.,s.1176,s.1235{linuxInterface}), Bash, 3-Tier-Architektur

% TODO: reformat chapter style with package titlesec
%\titleformat{\section}
%{\filcenter\normalfont\Large\bfseries}
%{\chaptertitlename~\thechapter} {0.5em} {}

% TODO: Bei erstem Vorkommen von Golang, REferenz auf deren Website setzen.

% TODO: Als abkürzungen festlegen und verlinken: HP, IBM, LXC

% TODO: ADD im Kapitel Dockerfile hinzufügen, da im Kapitel "Images" erwähnt.

% TODO: Isolierung-Unterkapitel in eine sinnvolle Reihenfolge bringen

% TODO: Eigenes Grundlagenkapitel? :
%         /proc Filesystem (evtl Auflistung der Files im Appendix) .... auch in \cite[S.42,S.223-227,S.231]{linuxInterface} und
%         DAC (normale Linux-Userauthentifizierung) .... auch in \cite(grundlagenkapitel){linuxInterface}
%             --> Users and Groups in \cite[S.153]{linuxInterface}, \cite[S.183]{linuxInterface}
%         System Calls (evtl Auflistung im Appendix) .... auch in \cite[S.43ff.,S.68]{linuxInterface}
%     PID (Process ID) ..... auch in \cite[S.114f.]{linuxInterface}

% TODO: Local DoS: Fork-bomb (siehe \cite[S.793]{linuxInterface})

% TODO: Unterschiedung von Produkt Docker und Unternehmen Docker. Das Unternehmen evtl. kursiv schreiben oder mit (TM)-Zeichen am Ende
          % --> Klaeren ob das okay so ist, auch obs ok ist das in "struktur der arbeit" zu schreiben

% TODO: Eigenes groses Kapitel "Docker im Vergleich zu anderen Containertechnologien"
%       Hauptsaechlich Docker und CoreOS' rkt vergleichen (rkt bietet KVM support an: man hat die wahl zwischen hardwarevirtualisierung und containern)



% \linespread{1.3}            % 1.5x line spacing
% \linespread{1.6}            % Double line spacing
% \hfill test                 % insert horizontal stretched space
% \vfill test                 % insert vertical stretched space
% ,,German quotation marks``  % deutsche Anfuehrungszeichen

%\subsection{Subsektion blabla}
%\subsubsection{Subsubsektion blablabla}
%\paragraph{Paragraph soundso}
%\subparagraph{SubParagraph soundso}
