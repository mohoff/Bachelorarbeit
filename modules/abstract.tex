\documentclass[../main.tex]{subfiles}
\begin{document}

% Abstract
\begin{abstract}
\label{abstract}
  \noindent \centerline{\textbf{Abstract}}
  \newline
  \newline
  \noindent
  In recent years, virtualization technologies have established themselves in server infrastructures. Not including hypervisor-based virtualization approaches, Docker, as a popular representative of container-based virtualization, has proved to be very successful in the market. Despite many attractive properties, container systems are considered questionable in terms of security due to their architecture. This paper analyzes built-in security models and mechanisms as well as cloud-centric integration capabilities of container technologies using the example of Docker.

  \vspace{1cm}
  \noindent \centerline{\textbf{Kurzfassung}}
  \newline
  \newline
  \noindent
  Virtualisierungstechnologien haben sich über die letzen Jahre in Serverinfrastrukturen etabliert. Neben hypervisorbasierten Techniken hat sich mit dem Erfolg von Docker die containerbasierte Virtualisierung auf dem Markt behauptet. Abgesehen von vielen attraktiven Eigenschaften von Containersystemen, ist die Sicherheit dieser aufgrund ihrer Architektur fragwürdig. Diese Arbeit untersucht die verwendeten Sicherheitsmodelle und -mechanismen sowie Integrationsmöglichkeiten in Cloud-Infrastrukturen von Containertechnologien am Beispiel von Docker.
\end{abstract}

\end{document}
