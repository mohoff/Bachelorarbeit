% GLOSSARY ENTRIES %

% Usage:
  % \gls{LABEL}
  % \Gls{LABEL}       % Erster Buchstabe ist Großbuchstabe
  % \glspl{LABEL}     % Pluralform, wenn definiert
  % \Glspl{LABEL}     % Pluralform und Großbuchstabe am Anfang
  % \glslink{LABEL}{ALTERNATE-TEXT}   % Setzt Glossarlink, aber alternativer Text
  % \


% Mandatory key-value-pairs: name, description
% Optional key-value-pairs: sort, plural, symbol

\newglossaryentry{Cloud}
{%
  name={Cloud},
  description={Eine entfernte Rechnerinfrastruktur, die Dienste (Anwendungen, Plattformen, etc.) zur Nutzung bereitstellt.
    \begin{itemize}
      \item Private Cloud: Dienste werden aus Gründen der Sicherheit oder des Datenschutzes nur firmenintern für eigenen Mitarbeiter angeboten.
      \item Public Cloud: Dienste sind öffentlich nutzbar.
      \item Hybrid Cloud: Mischform aus einer privaten und öffentlichen Cloud. Manche Dienste werden nur firmenintern verwendet, andere auch von außerhalb des Firmennetzes.
    \end{itemize}\cite{cloud}
  }
}

\newglossaryentry{MultiTenantService}
{%
  name={Multi-Tenant-Service},
  description={Eine Serveranwendungen, die mehrere Nutzer gleichzeitig verwenden. Jeder Nutzer kann nur auf seine eigenen Daten zugreifen und interferiert nicht mit anderen Nutzern. Auf dem Server kann die Anwendung, die dieses Prinzip umsetzt, in einer Instanz (ohne Redundanz) laufen \cite{multitenant}}
}

\newglossaryentry{BestPractice}
{%
  name={Best-Practice},
  description={Eine bestimmte, ideale Vorgehensweise für dem Umgang mit einer Sache, die zu einem erwünschten Zustand, z.B. der Erfüllung eines Standards, beiträgt. Im Fall von Docker kann es eine Best-Practice sein, Images zu signieren um deren Integrität zu gewährleisten}
}

\newglossaryentry{DevOps}
{%
  name={DevOps},
  description={DevOps-Teams sind sowohl für die Entwicklung (\emph{Dev} = Development) eines Produkts als auch den Betrieb (\emph{Ops} = Operations) dessen verantwortlich. Durch die gemeinsame Ergebnisverantwortung fällt der Overhead einer Übergabe, zwischen ansonsten getrennten Teams, weg \cite{devops}}
}
