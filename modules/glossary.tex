% GLOSSARY ENTRIES %

% Usage:
  % \gls{LABEL}
  % \Gls{LABEL}       % Erster Buchstabe ist Großbuchstabe
  % \glspl{LABEL}     % Pluralform, wenn definiert
  % \Glspl{LABEL}     % Pluralform und Großbuchstabe am Anfang
  % \glslink{LABEL}{ALTERNATE-TEXT}   % Setzt Glossarlink, aber alternativer Text
  % \


% Mandatory key-value-pairs: name, description
% Optional key-value-pairs: sort, plural, symbol

\newglossaryentry{Cloud}
{%
  name={Cloud},
  description={Eine Serverinfrastruktur, die Dienste (Anwendungen, Plattformen, etc.) zur Nutzung bereitstellt.
    \begin{itemize}
      \item Private Cloud: Dienste werden in einem eigenen Rechenzentrum betrieben.
      \item Public Cloud: Dienste werden in Rechenzentren externer Anbieter betrieben. Virtuelle Instanzen werden mit anderen Kunden des Betreibers geteilt. Beispiele: \emph{Amazon Web Services}, \emph{Microsoft Azure}
      \item Hybrid Cloud: Mischform aus Private und Public Cloud. Nahtlose, für den Nutzer unsichtbare Integration der beiden Grundformen.
    \end{itemize}
  }
}

\newglossaryentry{MultiTenantService}
{%
  name={Multi-Tenant-Service},
  description={Serveranwendungen bzw. -umgebung, die mehreren Nutzern (ggf. Kunden) gleichzeitig dient}
  %Auf dem Server kann die Anwendung, die dieses Prinzip umsetzt, in einer Instanz (ohne Redundanz) laufen \cite{multitenant}
}

\newglossaryentry{BestPractice}
{%
  name={Best-Practice},
  description={Bestimmte, ideale Vorgehensweise für dem Umgang mit einer Sache, die zu einem erwünschten Zustand, z.B. der Erfüllung eines Standards, beiträgt. Im Fall von Docker kann es eine Best-Practice sein, Images zu signieren um deren Integrität zu gewährleisten}
}

\newglossaryentry{DevOps}
{%
  name={DevOps},
  description={DevOps-Teams sind sowohl für die Entwicklung (\emph{Dev} = Development) eines Software-Produkts als auch den Betrieb (\emph{Ops} = Operations) dessen verantwortlich}
  % Durch die gemeinsame Ergebnisverantwortung fällt der Overhead einer Übergabe, zwischen ansonsten getrennten Teams, weg \cite{devops}}
}

\newglossaryentry{Build}
{%
  name={Build},
  description={Ein Erstellungsprozess, bei dem Quellcode in ein Objektcode bzw. direkt in ein fertiges Programm automatisch konvertiert wird}
}

\newglossaryentry{weicheHarteLimits}
{%
  name={weiche und harte Limits},
  description={Numerische Angabe zur maximalen Nutzung einer Ressource. Eine Methode, um Verfügbarkeit zu ermöglichen. Das weiche Limit dient als Richtwert. Das harte Limit stellt den Maximalwert dar. Angestrebeter Zustand: weiches Limit \textless= hartes Limit}
  % In der Implementierung in Solaris, startet bei Überschreitung des weichen Limits ein Timer. Wenn Timer eine bestimmte Zeit überschreitet, wird weiches Limit kurzzeitig wie das harte Limit erzwungen \cite{softHardLimits}}
}

% TODO: gescheite Buchquelle finden fuer DoS
\newglossaryentry{DenialOfService}
{%
  name={Denial of Service},
  description={Allgemeine Bezeichnung von Angriffen, die i.d.R. durch die Überlastung von Ressourcen das Sicherheitsziel der Verfügbarkeit verletzen. Beispiel: Fork-Bomb},
  see={ForkBomb}
}

\newglossaryentry{KernelObject}
{%
  name={Kernelobjekt},
  description={Datenstrukturen im Kernel, die verschiedene Ressourcen (z.B. Datei, Verzeichnis, etc.) abbildet und von LSMs ausgewertet werden kann \cite{kernelObjects}}
}

\newglossaryentry{chroot}
{%
  name={chroot},
  description={Steht für \texttt{change root} und ist eine Funktion von UNIX-Systemen, um das Root-Verzeichnis des aktuellen Prozesses zu wechseln \cite{chroot}. Grundlegendes Konzept für Linux-Namespaces}
}

\newglossaryentry{SystemCall}
{%
  name={System Call},
  description={Fundamentale Schnittstelle unter Linux, die Anwendungen den Zugriff auf Funktionen des Kernels ermöglicht \cite{systemcall}}
}

\newglossaryentry{CopyOnWrite}
{%
  name={Copy-On-Write},
  description={Verfahren, bei dem mehrere Prozesse auf die gleiche, geteilte Ressource zugreifen. Erst wenn ein Prozess die Ressource manipulieren will, erhält dieser eine eigene Kopie der Ressource. Somit ist die Integrität der Ressource für die anderen Prozesse weiterhin gewährleistet \cite{dockerImagesAndContainers}}
}

\newglossaryentry{Shell}
{%
  name={Shell},
  description={Benutzerschnittstelle, die interaktive und kommandobasierte Nutzung von Betriebssystemen und Anwendungen ermöglicht. Unter Linux meist als Bash verfügbar, eine Implementierung der Shell}
}

\newglossaryentry{ZeroDayExploit}
{%
  name={Zero-Day-Exploit},
  description={Schwachstelle in einer Software, die von Angreifern ausgenutzt werden kann, da sie den Erstellern der Software nicht bekannt ist.}
}

\newglossaryentry{ResponsibleDisclosure}
{%
  name={Responsible Disclosure},
  description={Bezeichnet in der IT eine spezielle Art der Bekanntmachung von Schwachstellen in einer Software. Dabei einigen sich alle Beteiligten darauf, i.d.R. auf Vertrauensbasis, die Schwachstelle einen bestimmten Zeitraum oder bis sie behoben wurde, geheim zu halten}
}

\newglossaryentry{DreiTierArchitektur}
{%
  name={3-Tier-Architektur},
  description={Serverarchitektur, die Teilung eines Dienstes in drei verbundene Schichten vorsieht: Anwendungs-, Logik- und Datenschicht}
}

\newglossaryentry{VirtualAppliance}
{%
  name={Virtual Appliance},
  description={VM, die Paket aus Anwendung, Bibliothek und Betriebssystem enthält. Kunden erwerben funktionierende VM, anstelle einer alleinigen Anwendung \cite[S.672f.]{tanenbaumOS}}
}

\newglossaryentry{ForkBomb}
{%
  name={Fork Bomb},
  description={Angriff, der die Ressourcen eines Systems erschöpft, indem ein das schadhafte Programm rekursiv Kopien von sich selbst in einer Endlosschleife erzeugt. Durch den damit verbundenen Speicherbedarf, verletzt der Angriff das Sicherheitsziel der Verfügbarkeit}
}

\newglossaryentry{ImmutableInfrastructure}
{%
  name={Immutable Infrastructure},
  description={Konzept für den Betrieb von serverseitiger Software. Verlangt, dass Dienste zustandslos operieren und im Betrieb nicht modifiziert werden dürfen. Dienste werden einmalig beim Startvorgang aus Vorlegen, z.B. einem Dockerfile, erstellt. Daten (Zustand) werden außerhalb besagter Dienste gehalten \cite{unikernelMeetsDocker}\cite{immutableInfrastructure}}
}
