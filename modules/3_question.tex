\documentclass[../main.tex]{subfiles}
\begin{document}

\chapter{Fragestellung}
\label{question}
  Die Wertschöpfung moderner IT-Unternehmen beruht auf dem Angebot von Diensten, auch Services genannt, die über das Internet den Nutzern zur Verfügung gestellt werden. Die Services werden von Anwendungen angeboten, die in der Regel selbst in Rechenzentren betrieben werden. Der überwiegende Vermögensgegenstand in diesem Modell ist die Software, die in den Rechenzentren produktiv läuft. Der Wert dieser ist direkt abhängig von der Funktionstüchtigkeit eines Rechenzentrums. Je nach Anwendungsfall kommt den Sicherheitszielen aus Kapitel \ref{introSecGoals} unterschiedliche Wichtigkeit zu.

  Kunden, die ihr Produkt über Rechenzentren anbieten sind u.a. an Sicherheitsfeatures interessiert, die die Sicherheitsziele der Vertraulichkeit, Integrität und Verfügbarkeit sicherstellen.
  Die Betreiber von Rechenzentren streben neben den zur Verfügung gestellten Sicherheitsfunktionen, einen möglichst effizienten Betrieb der Kundensysteme in der eigenen Infrastruktur an.

  Wie in Kapitel \ref{basics} gezeigt, bietet der Einsatz von Containertechnologien aus Sicht der Performance und Effizienz sowie administrativen Faktoren, attraktive Eigenschaften für die Betreiber von Rechenzentren. Mit gegebener Hardware lassen sich z.B. mit Containerlösungen i.d.R. mehr Kundenanwendungen realisieren, wie wenn letztere mit konventioneller Hypervisor-Technologie betrieben werden.

  % Die Betreiber von Rechenzentren wiederum müssen diese Nachfrage befriedigen.
  % Dabei streben sie einen möglichst effizienten
  %, wollen gleichzeitig jedoch auch selbst Gewinn machen. Durch den Betrieb von Containern in Rechenzentren kann im Vergleich zu Hypervisorlösungen theoretisch mehr Gewinn geschöpft werden, da sich mit ihnen - bei gleicher Hardware - mehr virtuelle Kapazität realisieren lässt. Die Sicherheit virtueller Instanzen darf im Betrieb von Containern jedoch nicht leiden, um weiterhin den Kundenanforderungen zu entsprechen.

  Mit den außer Frage stehenden Leistungs- und Effizienzvorteilen, ist demnach die erste zentrale Frage für Betreiber von Cloud-Infrastrukturen, inwiefern Sicherheitsfunktionen von Containertechnologien die erforderlichen Sicherheitsziele erfüllen. Diese Fragestellung kann anhand einer genaueren Untersuchung, angelehnt an die von Mandl in \cite[S.36]{CISSP} vorgestellte Risikoanalyse, genauer formuliert werden. Die Risikoanalyse dient gleichzeitig dazu, Annahmen vorzustellen und Schlussfolgerungen zu ziehen, auf deren Basis die zu betrachtenden Eigenschaften von Docker gegen Ende dieses Kapitels definiert werden.

  \section{Identifikation der Bedrohungen}
    Die verschiedenen Bedrohungsarten werdem im Folgenden auf Basis eines Systemmodells der Container-basierten Virtualisierung identifiziert.

    Das Systemmodell von Hypervisor-basierten Systemen kann nicht verwendet werden, da das Design der Containersysteme stark von Erstgenannten abweicht. Während nach \cite[S.125]{CISSP} virtuelle Maschinen als eigener Sicherheitsmechanismus des Betriebssystems aufgelistet ist, stimmt das für die Containersysteme und deren Architektur nicht. Andere Sicherheitsfeatures des Hosts, die ab Kapitel \ref{secLinux} vorgestellt werden, müssen aktiviert werden, um die Containersicherheit zu erhöhen.

    \emph{Bild zum Systemmodell}

    % TODO: Systemmodell erläutern ... Erst was abgebildet ist, dann mathematisch fundiert.
    Das Systemmodell ist zunächst definiert als ein Docker-Host h und ein Set C, das aus Containern C1, C2, ... Cn besteht. Diese Container laufen auf einem Docker-Host, der aus einem Betriebssystem besteht, auf dem ein Docker-Daemon läuft, der C verwaltet. Verteilte Docker-Systeme, in denen mehrere Hosts miteinander kommunizieren, werden in dieser Arbeit nicht betrachtet. Aus diesem Grund enthält eine Menge von Hosts H in diesem Fall nur einen Host h.

    Aus \fig \ref{fig:systemModel} und der Funktionsweise von Docker (siehe Kapitel \ref{dockerIntro} können zwei unterschiedliche Gefahrenquellen (A) und (B) identifiziert werden:

    \begin{enumerate}[label=(\Alph*)]
      \item \textbf{Docker-Spezifika:} In Containern werden Anwendungen ausgeführt, die nicht zwangsweise vertrauenswürdig sind. Wenn beispielweise Containerimages von einer öffentlichen Registry, wie dem Docker Hub, bezogen werden, existiert keine Garantie, dass aus diesen Images gestartete Container gegen keines der drei zuvor definierten Sicherheitsziele verstößt. Durch die Komplexität moderner Anwendungen und deren Abhängigkeiten zu Bibliotheken, ist es selbst bei quelloffenen Anwendungen schwierig, diese als vertrauenswürdig einzustufen. Deswegen muss davon ausgegangen werden, dass in Containern willkürliche Programme ablaufen (\emph{interne Gefahrenquellen}), die - versehentlich oder beabsichtigt - den Host schädigen können.

      Außerdem kann jederzeit ein Fehler in der Codebasis von Docker entdeckt werden, der die Sicherheit von Docker-Hosts und Containern beeinträchtigt.

      \item \textbf{Container als Server im Internet:} Viele Container stellen einen Dienst über das Internet zur Verfügung und stehen dadurch mit der Außenwelt in Kontakt. Ein Webserver beispielsweise, kann als Containerapplikation betrieben werden, indem er über über einen Port Anfragen von Clients entgegennimmt und diese nach abgeschlossener Verarbeitung beantwortet. Die Notwendigkeit, Containerschnittstellen über das Internet anzubieten, kann von Angreifern (\emph{externe Gefahrenquellen}) ausgenutzt werden, um Sicherheitsziele zu verletzen. Der Schutz des Netzwerks und der Verbindung des Hosts an das Internet, der in dieser Arbeit nicht behandelt wird, muss unabhängig von der eingesetzten Containertechnologie realisiert sein.
    \end{enumerate}

  \section{Auswirkungen der identifizierten Risiken:}
    Bei einer näheren Betrachtung, sind aus Sicht des Hostsystems die Folgen beider Gefahrenquellen jedoch sehr ähnlich. In beiden Fällen muss davon ausgegangen werden, dass ein Container schadhaften Code ausführt. Die Unterscheidung nach Gefahrenquelle (A) oder (B) ist unerheblich, da beide Szenarien dazu führen, dass eine oder mehrere Container eines Hostsystems schadhaften Code ausführen.

    Daraus ergibt sich die Schlussfolgerung, dass ein Container nicht vertrauenswürdig ist und aus Sicht der IT-Sicherheit angenommen werden muss, dass dieser von einem Angreifer kontrolliert wird.

    Mithilfe dieser Erkenntnis kann das Systemmodell unter Einbeziehung eines Angreifermodells erweitert werden. M wird in zwei echte Teilmengen unterteilt, die jeweils korrekt funktionierende, legitime Container und kompromittierte Container repräsentieren:

    C TEILMENGEVON M = {c1,c2,...ci | 0 < i < N UND C != {} UND i+j = N} mit c ELEMENTVON C     % ci == c
    C' TEILMENGEVON M = {c1,c2,...cj | 0 < j < N UND C' != {} UND i+j = N} mit c' ELEMENTVON C'    % cj == c'
    C' vereinigt mit C ergibt M.

    Mit 0 < i < N, 0 < j < N und {C,C'} != {} wird ein Systemmodell erzeugt, das von mindestens zwei Container in einem Hostsystem ausgeht (#{M} > 1), wovon mindestens ein Container c legitim (#{C} > 0) und mindestens ein Container c' kompromittiert (#{C'} > 0) ist.

    Dieses Modell bildet den Betrieb von Docker-Containern realitätsnah ab, da diese so konstruiert wurden, dass sie i.d.R. in einer Vielzahl parallel auf h laufen. Wie aus obiger Schlussfolgerung hervorgeht, muss angenommen werden, dass C' eine nicht-leere Menge ist, also mindestens ein Container c' existiert, der Schadcode ausführt. Dabei kann c' die Sicherheitsziele von h und C verletzen. Beide Eigenschaften werden von diesem Systemmodell abgedeckt.


    % Ob das zugrundeliegende Image fehlerhaft bzw. manipuliert ist (Gefahr A), oder ein Container aktiv von einem Angreifer kontrolliert wird (Gefahr B), spielt für den Host keine Rolle. Der Host muss in der Lage sein, die Systemsicherheit aufrecht zu erhalten. Die Systemsicherheit umfasst hierbei den Schutz des Hosts sowie anderer Container.

    (Per Defintion wird ein Container, der schadhaften Code ausführen kann, also bösartig bezeichnet. Ein korrekt funktionierender Container, dessen Sicherheitsziele aufrecht gehalten werden sollen, wird abgekürzt als legitim aufgeführt.)

    \emph{TODO: Formale Defintion: Set an Containern C auf einem Host. Annahme ist, dass nicht-leeres Subset C\' auf dem Host existiert, das bösartig ist. C\' kann 1 bis \#\{C\} groß sein. Wenn C'\ maximal \#\{C\}-1 groß ist, führt das zu einer stärker Behauptung, da es hierbei auch min. 1 legitimen Container zu schützen gibt}
    % TODO:Formale definition

    Container c' aus dem Set C' ist in der Lage alle drei Sicherheitsziele zu verletzen. Man-in-the-Middle kann Vertraulichkeit verletzt werden, indem geheime Informationen abgefangen werden. Mit geheimen Informationen können unter Umständen Daten unrechtmäßig manipuliert werden, was die Integrität beeinflusst. Normale Programmflüsse können unterbrochen werden, was Beeinträchtigungen für die Verfügbarkeit mit sich zieht. Auch DoS-Attacken sind von c' aus möglich.

    Einige der von c' geführten Angriffe sind nur durchführbar, wenn der Container im Besitz bestimmter Rechte ist. Die Privilegien, die ein Container standardmäßig besitzt, können fest definiert werden.

    \emph{Privilege Escalation als extra Punkt aufführen? Ist eigtl kein direktes Sicherheitsziel. Eher im Punkt Gegenmaßnahmen aufführen....}

    \begin{itemize}
      \item \textbf{Vertraulichkeit}: c' kann Aktionen ausführen, die unbefugten Zugriff auf Informationen des Hosts oder anderer Container gewähren.
      \item \textbf{Integrität}: c' kann Aktionen ausführen, die Daten des Hosts oder anderer Container manipulieren.
      \item \textbf{Verfügbarkeit}: c' kann Aktionen ausführen, um von dem Host vorgesehene Mechanismen zur Aufrechterhaltung der Verfügbarkeit zu umgehen. Die Funktionstüchtigkeit des Hosts und die anderer Container kann dadurch beeinträchtigt werden.
    \end{itemize}

    %- Container compromise: compromise C k ∈ C by means of illegitimate
    %data access, Man-in-the-Middle (MitM) attacks or by affecting the control
    %flow of instructions executed in C k ∈ C.
    %- Denial of Service: disturb normal operation of the host or C k ∈ C.
    %- Privilege escalation: obtain a privilege not originally granted to a C j ∈ C.

  \section{Anforderungen an die Sicherheit von Containern}
    Aus den Annahmen und der Schlussfolgerung im letzen Abschnitt, lassen sich allgemeine Sicherheitsanforderungen formulieren:

    Allgemeine Anforderung: Von einem kompromittiertem Container c', darf der Rest des Systems nicht betroffen sein. Die Sicherheitsziele der Vertraulichkeit, Integrität und Verfügbarkeit sollen für das Hostsystem und andere Container gewahrt werden.

  \section{Theoretischer Lösungsansatz}
    Die Strategie zur Erfüllung der soeben definierten Sicherheitsanforderungen, beruht unter Docker auf einem zweistufigen Ansatz:

    \begin{enumerate}[label=(\arabic*)]
      \item Angreifern soll es erschwert oder bestenfalls unmöglich gemacht werden, aus der virtualisierten Umgebung eines c' auszubrechen. Angreifer sollen nicht die Möglichkeit besitzen, die ursprünglich für c' vorgesehenen Rechte zu erweitern.
      \item Falls es einem Angreifer dennoch gelingen sollte, (1) zu verletzen, soll er durch weitere Zugriffskontrollen daran gehindert werden, Schaden an h und C anzurichten.
    \end{enumerate}

    Mithilfe der im nächsten Abschnitt aufgelisteten Mittel, versucht Docker diese umzusetzen.

    %auf Modellebene:
    %1. + 2. setzen \emph{Defense in Depth} um.
    %2. setzt \emph{Principle of Least Privilege} um.

  \section{Umsetzung des Lösungsansatzes} % um Bedrohungspotential zu minimieren.
    Zunächst lassen sich die möglichen Sicherheitsmechanismen in drei Arten unterteilen. Zum Verständnis des Aufbaus der folgenden Kapitel, sind diese kurz vorgestellt \cite[S.40]{CISSP}:

    \begin{itemize}
      \item \textbf{Technische Kontrollen:} Umfasst alle hardware- und softwarebasierten Mechanismen, z.B. ein Zugriffsschutz unter Verwendung einer MAC.
      \item \textbf{Administrative Kontrollen:} Enthält Management-Kontrollen, die z.B. durch Konfigurationen, Entwicklung einer Sicherheitspolitik, Best Practices, Sicherheitsschulungen des Personals, umgesetzt werden.
      \item \textbf{Physische Kontrollen:} Beinhalten Mechanismen wie Sicherheitsschleusen, Schlösser und Wachpersonal. Obwohl ein Bezug zum Betrieb von Rechenzentren hergestellt werden kann, haben physische Kontrollen keine spezifische Relevanz für die containerbasierte Virtualisierung und sind aus diesem Grund an dieser Stelle nur zum Zweck der Vollständigkeit aufgeführt.
    \end{itemize}

    Docker hat als Softwareprodukt die Möglichkeit, direkt technische Kontrollen in die Codebasis einzubauen bzw. solche, die der Linux-Kernel implementiert, zu unterstützen. Auch administrative Sicherheitsansätze verfolgt Docker, die teilweise durch technische Mittel anwendbar sind.

  \section{Ziel der Arbeit}
    Gegenstand dieser Arbeit ist, die von Docker integrierten technischen, Software-basierten sowie administrativen Maßnahmen zu analysieren.

    Im Rahmen der Arbeit werden die Sicherheitsmaßnahmen sowie deren Einstatz unter Docker vorgestellt. Die in diesem Kapitel erarbeiteten Annahmen und Schlussfolgerungen bilden dabei die Basis der Untersuchung. Die einzelnen Sicherheitsmerkmale sind dabei unabhängig von der Infrastruktur. Das bedeutet, dass sie bei einem lokalen Betrieb von Docker auf Computern von Entwicklern, bis hinzu Docker-Installationen in Cloud-Infrastrukturen relevant sind.

    In dieser Arbeit wird auch betrachtet, welche Integrationsmöglichkeiten für Docker in Cloud-Infrastrukturen existieren und welche speziellen Sicherheitskomponenten einsetzbar sind bzw. von externen Dienstleistern angeboten werden.



  %Docker setzt auf Software-Ebene überwiegend technische Kontrollen um, die kombiniert eine mehrschichtige \emph{Defense In Depth} bilden.
  % TODO: Docker als Softwareprodukt setzt überwiegend technische Kontrollen um (Kapitel 4). Aber begünstigt auch adminstrative Kontrollen (Kapitel 5).

  % TODO: Entweder in Fazit oder Overview-Kapitel
  Docker verfolgt eine softwarebasierte \emph{Defense In Depth}, bei der verschiedenartige Sicherheitsschichten realisiert werden, um einen bestmöglichsten Schutz zu ermöglichen. Eine Geheimhaltung von technischen Sicherheitsmaßnahmen, auch \emph{Security Through Obscurity} genannt, kann nicht praktiziert werden, da Docker und Linux selbst quelloffene Projekte sind.

  % TODO: Besser als Intro in Kapitel 4
  Zunächst sind die grundlegenden Mechanismen zur Isolation und Ressourcenverwaltung zu erwähnen, die das Konzept von Containern ermöglichen. Unter Linux sind das die Techniken Namespaces für die Isolation und Control Groups für die Ressourcenverwaltung, die Docker und andere Linux-basierten Containertechnologien nutzen. Während beide Mechanismen eine hohe Sicherheitsrelevanz aufweisen, basiert deren Design auf der Annahme, dass alle Container sowie das Hostsystem legitim sind. Beide Mechanismen können als Feature-erweiternde Hostfunktionen verstanden werden, deren alleiniges Ziel es ist, die Idee von virtuellen Gastsystemen (Containern) auf der Basis des \texttt{chroot}-Befehls zu realisieren. Die in CISSP definierten sicherheitserhöhenden Maßnahmen werden von Namespaces und Cgroups nicht berührt.

  % TODO: besser als Intro in Kapitel 5
  Auf Basis der in \ref{...} erarbeiteten Annahme, dass Angreifer ausgehend von Containern Angriffe gegen die Sicherheitsiele ausüben kann, müssen weitere Sicherheitsmechanismen verwendet werden, um das Hostsystem sowie andere legitime Container zu schützen. Diese zusätzlichen Mechanismen realisieren Zugriffskontrollen, die auch dann den sicheren Zustand von Hostsystemen und anderer Container erhalten, wenn es einem Angreifer möglich ist, seine Privilegien unrechtmäßig auszudehnen.

  % Angewandt auf die Praxis: es muss universeller Ansatz gewählt werden, da jeder Kunde andere Sicherheitsanforderungen hat. Demnach können Schwachstellen, die die Vertraulichkeit, die Integrität oder Verfügbarkeit der Kundensoftware bedrohen, fatale Folgen für den Umsatz und die Reputation der Kunden und Betreiber von Rechenzentren ergeben.

  % Welche Sicherheitsmodelle und -mechanismen können eingesetzt werden, um Bedrohungspotential von aufgeführten Gefahrenquellen zu minimieren.

  % Darunter fallen mit Software realisierte Mechanismen zur Isolation, Ressourcenverwaltung und Zugriffskontrollen
  % Ziel der Arbeit: Halten diese CIA-Triade ein? Wie effektiv/weitrechend sind die eingesetzten modelle/mechanismen. Untersuchung der -modelle und -mechanismen in Bezug auf CIA-Traide.



  % Kommt man von Container auf Host-OS? Von Container auf anderen Container? ~etc.
  % 5-6 Sicherheitsziele erwähnen. Mit Forschungsfrage in Bezug bringen --> später bei Isolierung und Ressourcenverwaltung wieder aufgreifen

  % Allgemeiner Überblick zu Docker Security in \cite[S.3]{dockerSeec1}

  % (TODO): Schreiben, dass Container-Verschatelung bzw. Container in Hierarchien  ins Systemmodell absichtlich nicht aufgenommen wurden (was aber in FreeBSD jails moelgich ist).
  %   ^   (vgl. beide  \cite[S.4]{dockerSec2})

  % TODO: Sicherstellen, das Abgrenzung zu Netzwerkthemen da ist.

  %Das zentrale Konzept, auf dem alle Containertechnologien beruhen, ist das der Isolierung. Im Kontext von Containern kann die Isolierung definiert werden als Trennung zwischen Containern und einem Host, sowie die Trennung zwischen Containern \cite[S.1]{dockerSec2}.

  %Auf einem System mit Host und einem oder mehreren Containern, stellt sich zunächst die Frage welche Art und Richtung von Kommunikation zwischen diesen beiden Komponenten erlaubt und nicht erlaubt sein soll. Dadurch, dass der Docker-Daemon auf dem Host läuft und es dessen Aufgabe ist u.a. den Container-Lifecycle zu kontrollieren, braucht dieser Zugriff auf die Container. Verallgemeinert ist also die Kommunikation von Host zu Container erforderlich und damit erlaubt.

  %Was in einem Container passiert, ist zweitrangig, da der Container bei Fehlfunkionen jederzeit seitens des Hosts neu gestartet werden kann. Wichtig ist aber, dass der Container selbst von der Außenwelt, also dem Host und anderen Container, isoliert ist und seine Aufrufe gegen den Hostkernel streng limitiert sind und diese den Host nicht beeinträchtigen können.

  %Mehrere Sicherheitsfragen für Container-basierte Systeme sind in den folgenden Punkten formuliert. Sie beruhen auf der Annahme, dass ein Angreifer die Kontrolle über einen Container X übernommen hat und versucht, über diesen Schaden zu verursachen.

  %Situationen, in denen ein Angreifer bereits zu Beginn die Kontrolle über den Host hat, werden nicht betrachtet, da der Angreifer in dieser Lage bereits gewonnen hat und Container nach belieben manipulieren kann.

  %\begin{enumerate}[(1)]
  %  \item Ist es dem Angreifer möglich, seine in X erworbenen Rechte auf den Hosts zu erweitern, sodass er auf letzteren Root-Rechte erwirken kann? (Verletzte Sicherheitsziele: Vertraulichkeit, Authenzität, Integrität)
  %  \item Ist es dem Angreifer möglich, auf einen anderen Container Y des gleichen Hosts zuzugreifen? (Verletzte Sicherheitsziele: Vertraulichkeit, Authenzität, Integrität)
  %  \item Ist es dem Angreifer möglich, den Container oder Host auf eine Art und Weise zu beeinflussen, die den Betrieb anderer Container auf diesem oder entfernten Hosts beeinträchtigt? (Verletzte Schutzziele: Verfügbarkeit, Integrität) (Ressourcenverwaltung)
  %  \item Ist es dem Angreifer möglich, den Container X negativ zu beeinflussen oder ihn zum Absturz zu bringen? (Lifecyclemanagement des Docker-Hosts)
  %  \item Wie wird natürlichen Fehlfunktionen von Containern entgegengewirkt? (Lifecyclemanagement des Docker-Hosts)
          % Also wenn plötzlich Exception auftritt und Containeranwendung abstürzt.
  %  \item \emph{weitere Punkte?}
  %\end{enumerate}

  %Frage (1.) und (2.) zielen auf technischer Ebene auf die Isolation der Container ab. Eine Umformulierung in \glqq{}Sind Container ausreichend isoliert, um den Host zu schützen?\grqq{} ist möglich.

  %Wenn von der Netzwerkseite abgesehen wird, lässt sich das Szenario der Fragestellung (2.) auf das der Frage (1.) reduzieren, da der Zugriff auf andere Container nur über den lokalen Host möglich ist. Genauer gesagt ist der Zugriff auf andere Prozesse nur dann möglich, wenn Root-Rechte auf dem Host vorhanden sind. Die bereits generalisierten Sicherheitsfrage ist in (A.) unter Berücksichtigung dieses Punkts, erweitert

  %Die Fragen (3.), (4.) und (5.) teilen sich den Aspekt der Verfügbarkeit, der in Formulierung (B.) aufgegriffen wird.

  %Finale Umformulierungen und Generalisierungen:

  %\begin{enumerate}[(A)]
  %  \item Sind Container ausreichend isoliert, sodass ausgehend von Containern keine Root-Rechte auf dem Hostsystem erwirkt werden können?
  %  \item Kann der Betrieb von Containern negativ beeinflusst werden, sodass die Verfügbarkeit von Anwendungen darunter leidet?
  %  \item \emph{weitere Punkte?}
  %\end{enumerate}

  %, ob aus Containern ausgebrochen werden kann (1) und ob von einem Container X Zugriff auf einen Container Y auf dem gleichen Host möglich ist (2). Ist es unter der Annahme, dass ein Angreifer Zugriff auf einen Container erlangt, möglich, seine Rechte zu eskalieren und Root-Rechte auf dem Host zu erlangen? Diese Herausforderung spielt auch bei Frage (2) die entscheidene Rolle, da ein Zugriff auf andere Container erst mit einer Kompromitierung des Hosts möglichst ist. Aus diesem Grund genügt es, Frage (1) zu beantworten, da durch eine Antwort der selben auch die erweiterte Fragestellung (2) zufriedenstellend beantwortet werden kann.

  % Von entscheidender Bedeutung ist diese Frage, weil ihre Beantwortung die Existenzgrundlage für Container und damit auch Docker bildet. Gäbe es keine Methodeberuht. Wäre die Frage im Produkt Docker nicht beantwortet,

  %\emph{ALT:}


  % Technische Sicherheit
  %Um Frage (1.) zu beantworten, wird im ersten Hauptkapitel die intrinsische Sicherheit von Docker untersucht. Damit ist eine Reihe von Sicherheitsfeatures des Linux Kernels gemeint, die u.a. Docker nutzt, um nach Aussage von \emph{Docker} sichere Container zu ermöglichen. Genauer werden Mechanismen zur Isolation, Ressourcen- und Zugriffsverwaltung betrachtet, da sie direkt mit den erwünschten Sicherheitszielen aus Kapitel \ref{introSecGoals} in Bezug stehen.

  %Des Weiteren stellt sich die Frage, ob die Arbeit mit Docker und seinen Containern sicher ist. Wie in der \hyperref[dockerIntro]{Einführung zu Docker} beschrieben, stellt Docker zusammen mit anderen Anbietern einen Workflow und eine Palette an Tools zur Verfügung, die die Arbeit mit Containern erleichtern sollen. Wie diese Tools zur Sicherheit bzw. Angreifbarkeit von Docker-Systemen beitragen, wird im Kontext von den Sicherheitszielen betrachtet.
  % Technische und administrative Sicherheit

  %Nicht betrachtet werden die Sicherheitsrisiken, die sich durch den Betrieb eines Rechnernetzwerks ergeben, in dem Docker-Knoten existieren. Sicherheit aus Sicht der Netzwerktechnik und den verschiedenen \acrshort{OSI}-Schichten ist nicht Gegenstand der Untersuchung.

  % TODO: einarbeiten: Siehe \cite[S.3]{virtVSContainer} ... rechts oben ..

  % Da Images aud dem Docker Hub malicious sein koennen ,,,, arbitray code ausgefuehrt wird ...
  % kann auch ein Container, der diesen Code ausfuehrt, nicht vertraut werden /--> kann malicious werden (boesartig, boese absichten verfolgen)

\end{document}
