\documentclass[../main.tex]{subfiles}
\begin{document}

\chapter{Fragestellungen / Problemformulierung}
\label{question}
  Die Wertschöpfung moderner IT-Unternehmen beruht auf dem Angebot von Diensten, auch Services genannt, die über das Internet den Nutzern zur Verfügung gestellt werden. Die Services werden von Anwendungen angeboten, die selbst in Rechenzentren betrieben werden. Der überwiegende Vermögensgegenstand in diesem Modell ist die Software, die in den Rechenzentren produktiv läuft. Der Wert dieser ist direkt abhängig von der Funktionstüchtigkeit eines Rechenzentrums. Je nach Anwendungsfall kommt den Sicherheitszielen aus Kapitel \ref{secGoals} unterschiedliche Wichtigkeit zu.

  Kunden, die ihr Produkt über Rechenzentren anbieten sind an Sicherheitsfeatures interessiert, die die drei Sicherheitsziele sicherstellen. Die Betreiber von Rechenzentren wiederum müssen diese Nachfrage befriedigen, wollen jedoch gleichzeitig auch selbst Gewinn machen. Durch den Betrieb von Containern in Rechenzentren kann im Vergleich zu Hypervisorlösungen theoretisch mehr Gewinn geschöpft werden, da sich mit ihnen - bei gleicher Hardware - mehr virtuelle Kapazität realisieren lässt. Die Sicherheit virtueller Instanzen darf im Betrieb von Containern jedoch nicht leiden, um weiterhin den Kundenanforderungen zu entsprechen.

  Die erste zentrale Frage ist demnach, in wie weit Container für Kundenservices in Rechenzentren Sicherheit bieten. Diese stark verallgemeinerte Fragestellung kann anhand einer Risikoanalyse, wie sie z.B. von Mandl in \cite[S.36]{CISSP} vorgeschlagen ist, genauer formuliert werden. Die Risikoanalyse dient gleichzeitig dazu, Annahmen vorzustellen und Schlussfolgerungen zu ziehen, auf deren Basis gegen Ende des Kapitels der Untersuchungsgegenstand der vorliegenden Arbeit definiert und abgegrenzt ist.

  \textbf{Identifikation der Vermögensgegenstände / Wertschöpfungsmerkmale}: Für Betreiber von Rechenzentren der sichere und zuverlässige Betrieb von Kundensoftware. Gewährleistung von Vertraulichkeit, Integrität und Verfügbarkeit der Daten und Anwendungen allen Kunden haben für den Betreiber höchste Priorität.

  \textbf{Identifikation der Bedrohungen}: Die verschiedenen Bedrohungsarten können auf Basis eines Systemmodells von containerbasierter Virtualisierung identifiziert werden.

  Das Systemmodell von Hypervisorsystemen kann nicht verwendet werden, da das Design der Containersysteme stark von Erstgenannten abweicht. Während nach \cite[S.125]{CISSP} virtuelle Maschinen als eigener Sicherheitsmechanismus des Betriebssystems aufgelistet ist, stimmt das für die Containersysteme und deren Konzeption nicht mehr. Andere Sicherheitsfeatures des Hosts, die ab Kapitel \ref{secLinux} vorgestellt werden, müssen aktiviert werden, um die Containersicherheit zu erhöhen.

  \emph{Bild zum Systemmodell}

  Bei der ersten Betrachtung kann von zwei unterschiedlichen Gefahrenquellen gesprochen werden.

  \textbf{A: Ausführung von unkontrollierbaren Images:} In Containern werden Anwendungen ausgeführt, die nicht zwangsweise vertrauenswürdig sind. Wenn beispielweise Containerimages von einem öffentlichen Hub bezogen werden, existiert keine Garantie, dass aus diesen Images gestartete Container gegen keines der drei zuvor definierten Sicherheitsziele verstößt. Durch die Komplexität moderner Anwendungen und deren Abhängigkeiten zu Bibliotheken, ist es selbst bei quelloffenen Anwendungen schwierig, diese als vertrauenswürdig einzustufen. Deswegen muss davon ausgegangen werden, dass in Containern willkürliche Programme ablaufen (\emph{interne Gefahrenquellen}), die - versehentlich oder beabsichtigt - den Host schädigen können.

  \textbf{B: Container als Server im Internet:} Viele Container stellen einen Dienst über das Internet zur Verfügung und stehen dadurch mit der Außenwelt in Kontakt. Ein Webserver beispielsweise, kann als Containerapplikation betrieben werden, indem er über über einen Port Anfragen von Clients entgegennimmt und diese nach abgeschlossener Verarbeitung beantwortet. Die Notwendigkeit, Containerschnittstellen über das Internet anzubieten, kann von Angreifern (\emph{externe Gefahrenquellen}) ausgenutzt werden, um Sicherheitsziele zu verletzen. Der Schutz des Netzwerks und der Verbindung des Hosts an das Internet muss unabhängig von der eingesetzten Containertechnologie realisiert werden.
  % TODO: fällt in die Netzwerksicherheit und ist damit nur Randthema dieser Arbeit


  \textbf{Auswirkungen potenzieller Risiken:}
  Bei einer näheren Betrachtung jedoch, sind aus Sicht des Hostsystems die Folgen beider Gefahrenquellen identisch. In beiden Fällen muss davon ausgegangen werden, dass ein Container schadhaften Code ausführt. Ob das zugrundeliegende Image fehlerhaft bzw. manipuliert ist (Gefahr A), oder ein Container aktiv von einem Angreifer kontrolliert wird (Gefahr B), spielt für den Host keine Rolle. Der Host muss in der Lage sein, die Systemsicherheit aufrecht zu erhalten. Die Systemsicherheit umfasst hierbei den Schutz des Hosts sowie anderer Container.

  Per Defintion wird ein Container, der schadhaften Code ausführen kann, also bösartig bezeichnet. Ein korrekt funktionierender Container, dessen Sicherheitsziele aufrecht gehalten werden sollen, wird abgekürzt als legitim aufgeführt.

  \emph{TODO: Formale Defintion: Set an Containern C auf einem Host. Annahme ist, dass nicht-leeres Subset C\' auf dem Host existiert, das bösartig ist. C\' kann 1 bis \#\{C\} groß sein. Wenn C'\ maximal \#\{C\}-1 groß ist, führt das zu einer stärker Behauptung, da es hierbei auch min. 1 legitimen Container zu schützen gibt}
  % TODO:Formale definition

  Container c' aus dem Set C' ist in der Lage alle drei Sicherheitsziele zu verletzen. Man-in-the-Middle kann Vertraulichkeit verletzt werden, indem geheime Informationen abgefangen werden. Mit geheimen Informationen können unter Umständen Daten unrechtmäßig manipuliert werden, was die Integrität beeinflusst. Normale Programmflüsse können unterbrochen werden, was Beeinträchtigungen für die Verfügbarkeit mit sich zieht. Auch DoS-Attacken sind von c' aus möglich.

  Einige der von c' geführten Angriffe sind nur durchführbar, wenn der Container im Besitz bestimmter Rechte ist. Die Privilegien, die ein Container standardmäßig besitzt, können fest definiert werden.

  \emph{Privilege Escalation als extra Punkt aufführen? Ist eigtl kein direktes Sicherheitsziel. Eher im Punkt Gegenmaßnahmen aufführen....}

  %- Container compromise: compromise C k ∈ C by means of illegitimate
  %data access, Man-in-the-Middle (MitM) attacks or by affecting the control
  %flow of instructions executed in C k ∈ C.
  %- Denial of Service: disturb normal operation of the host or C k ∈ C.
  %- Privilege escalation: obtain a privilege not originally granted to a C j ∈ C.

  \textbf{Schutzmechanismen / Gegenmaßnahmen:} % um Bedrohungspotential zu minimieren.
  Angewandt auf die Praxis: es muss universeller Ansatz gewählt werden, da jeder Kunde andere Sicherheitsanforderungen hat. Demnach können Schwachstellen, die die Vertraulichkeit, die Integrität oder Verfügbarkeit der Kundensoftware bedrohen, fatale Folgen für den Umsatz und die Reputation der Kunden und Betreiber von Rechenzentren ergeben.

  Welche Sicherheitsmodelle und -mechanismen können eingesetzt werden, um Bedrohungspotential von aufgeführten Gefahrenquellen zu minimieren.

  Darunter fallen mit Software realisierte Mechanismen zur Isolation, Ressourcenverwaltung und Zugriffskontrollen
  % Ziel der Arbeit: Halten diese CIA-Triade ein? Wie effektiv/weitrechend sind die eingesetzten modelle/mechanismen. Untersuchung der -modelle und -mechanismen in Bezug auf CIA-Traide.
  Einteilung der Kontrollmechanismen in administrative, technische und physische Kontrollen \cite[S.40]{CISSP}:

  \begin{itemize}
    \item \textbf{Administrative Kontrollen:} Enthält Management-Kontrollen, die z.B. durch Konfigurationen, Entwicklung einer Sicherheitspolitik, Best Practices, Sicherheitsschulungen des Personals, umgesetzt werden.
    \item \textbf{Technische Kontrollen:} Umfasst alle hardware- und softwarebasierten Mechanismen, z.B. ein Zugriffsschutz unter Verwendung einer DAC oder MAC. Docker verfolgt eine softwarebasierte \emph{Defense in depth}, bei der verschiedenartige Sicherheitsschichten realisiert werden, um einen bestmöglichsten Schutz zu ermöglichen. Eine Geheimhaltung von technischen Kontrollen, auch \emph{Security through obscurity} genannt, kann nicht praktiziert werden, da Docker und Linux selbst quelloffene Projekte sind.
    \item \textbf{Physische Kontrollen:} Beinhalten Mechanismen wie Sicherheitsschleusen, Schlösser und Wachpersonal. Obwohl ein Bezug zum Betrieb von Rechenzentren hergestellt werden kann, haben physische Kontrollen keine spezifische Relevanz für die containerbasierte Virtualisierung und sind aus diesem Grund an dieser Stelle nur zum Zweck der Vollständigkeit aufgeführt.
  \end{itemize}

  Die vorliegende Arbeit konzentriert sich auf die technischen, softwarebasierten Kontrollen, die von Docker eingesetzt werden. Aber auch administrative Methoden werden in Kapitel \ref{secEcosystem} vorgestellt.
  % Kommt man von Container auf Host-OS? Von Container auf anderen Container? ~etc.
  % 5-6 Sicherheitsziele erwähnen. Mit Forschungsfrage in Bezug bringen --> später bei Isolierung und Ressourcenverwaltung wieder aufgreifen

  % Allgemeiner Überblick zu Docker Security in \cite[S.3]{dockerSeec1}

  % (TODO): Schreiben, dass Container-Verschatelung bzw. Container in Hierarchien  ins Systemmodell absichtlich nicht aufgenommen wurden (was aber in FreeBSD jails moelgich ist).
  %   ^   (vgl. beide  \cite[S.4]{dockerSec2})

  % TODO: Sicherstellen, das Abgrenzung zu Netzwerkthemen da ist.

  %Das zentrale Konzept, auf dem alle Containertechnologien beruhen, ist das der Isolierung. Im Kontext von Containern kann die Isolierung definiert werden als Trennung zwischen Containern und einem Host, sowie die Trennung zwischen Containern \cite[S.1]{dockerSec2}.

  %Auf einem System mit Host und einem oder mehreren Containern, stellt sich zunächst die Frage welche Art und Richtung von Kommunikation zwischen diesen beiden Komponenten erlaubt und nicht erlaubt sein soll. Dadurch, dass der Docker-Daemon auf dem Host läuft und es dessen Aufgabe ist u.a. den Container-Lifecycle zu kontrollieren, braucht dieser Zugriff auf die Container. Verallgemeinert ist also die Kommunikation von Host zu Container erforderlich und damit erlaubt.

  %Was in einem Container passiert, ist zweitrangig, da der Container bei Fehlfunkionen jederzeit seitens des Hosts neu gestartet werden kann. Wichtig ist aber, dass der Container selbst von der Außenwelt, also dem Host und anderen Container, isoliert ist und seine Aufrufe gegen den Hostkernel streng limitiert sind und diese den Host nicht beeinträchtigen können.

  %Mehrere Sicherheitsfragen für Container-basierte Systeme sind in den folgenden Punkten formuliert. Sie beruhen auf der Annahme, dass ein Angreifer die Kontrolle über einen Container X übernommen hat und versucht, über diesen Schaden zu verursachen.

  %Situationen, in denen ein Angreifer bereits zu Beginn die Kontrolle über den Host hat, werden nicht betrachtet, da der Angreifer in dieser Lage bereits gewonnen hat und Container nach belieben manipulieren kann.

  %\begin{enumerate}[(1)]
  %  \item Ist es dem Angreifer möglich, seine in X erworbenen Rechte auf den Hosts zu erweitern, sodass er auf letzteren Root-Rechte erwirken kann? (Verletzte Sicherheitsziele: Vertraulichkeit, Authenzität, Integrität)
  %  \item Ist es dem Angreifer möglich, auf einen anderen Container Y des gleichen Hosts zuzugreifen? (Verletzte Sicherheitsziele: Vertraulichkeit, Authenzität, Integrität)
  %  \item Ist es dem Angreifer möglich, den Container oder Host auf eine Art und Weise zu beeinflussen, die den Betrieb anderer Container auf diesem oder entfernten Hosts beeinträchtigt? (Verletzte Schutzziele: Verfügbarkeit, Integrität) (Ressourcenverwaltung)
  %  \item Ist es dem Angreifer möglich, den Container X negativ zu beeinflussen oder ihn zum Absturz zu bringen? (Lifecyclemanagement des Docker-Hosts)
  %  \item Wie wird natürlichen Fehlfunktionen von Containern entgegengewirkt? (Lifecyclemanagement des Docker-Hosts)
          % Also wenn plötzlich Exception auftritt und Containeranwendung abstürzt.
  %  \item \emph{weitere Punkte?}
  %\end{enumerate}

  %Frage (1.) und (2.) zielen auf technischer Ebene auf die Isolation der Container ab. Eine Umformulierung in \glqq{}Sind Container ausreichend isoliert, um den Host zu schützen?\grqq{} ist möglich.

  %Wenn von der Netzwerkseite abgesehen wird, lässt sich das Szenario der Fragestellung (2.) auf das der Frage (1.) reduzieren, da der Zugriff auf andere Container nur über den lokalen Host möglich ist. Genauer gesagt ist der Zugriff auf andere Prozesse nur dann möglich, wenn Root-Rechte auf dem Host vorhanden sind. Die bereits generalisierten Sicherheitsfrage ist in (A.) unter Berücksichtigung dieses Punkts, erweitert

  %Die Fragen (3.), (4.) und (5.) teilen sich den Aspekt der Verfügbarkeit, der in Formulierung (B.) aufgegriffen wird.

  %Finale Umformulierungen und Generalisierungen:

  %\begin{enumerate}[(A)]
  %  \item Sind Container ausreichend isoliert, sodass ausgehend von Containern keine Root-Rechte auf dem Hostsystem erwirkt werden können?
  %  \item Kann der Betrieb von Containern negativ beeinflusst werden, sodass die Verfügbarkeit von Anwendungen darunter leidet?
  %  \item \emph{weitere Punkte?}
  %\end{enumerate}

  %, ob aus Containern ausgebrochen werden kann (1) und ob von einem Container X Zugriff auf einen Container Y auf dem gleichen Host möglich ist (2). Ist es unter der Annahme, dass ein Angreifer Zugriff auf einen Container erlangt, möglich, seine Rechte zu eskalieren und Root-Rechte auf dem Host zu erlangen? Diese Herausforderung spielt auch bei Frage (2) die entscheidene Rolle, da ein Zugriff auf andere Container erst mit einer Kompromitierung des Hosts möglichst ist. Aus diesem Grund genügt es, Frage (1) zu beantworten, da durch eine Antwort der selben auch die erweiterte Fragestellung (2) zufriedenstellend beantwortet werden kann.

  % Von entscheidender Bedeutung ist diese Frage, weil ihre Beantwortung die Existenzgrundlage für Container und damit auch Docker bildet. Gäbe es keine Methodeberuht. Wäre die Frage im Produkt Docker nicht beantwortet,

  %\emph{ALT:}


  % Technische Sicherheit
  %Um Frage (1.) zu beantworten, wird im ersten Hauptkapitel die intrinsische Sicherheit von Docker untersucht. Damit ist eine Reihe von Sicherheitsfeatures des Linux Kernels gemeint, die u.a. Docker nutzt, um nach Aussage von \emph{Docker} sichere Container zu ermöglichen. Genauer werden Mechanismen zur Isolation, Ressourcen- und Zugriffsverwaltung betrachtet, da sie direkt mit den erwünschten Sicherheitszielen aus Kapitel \ref{introSecGoals} in Bezug stehen.

  %Des Weiteren stellt sich die Frage, ob die Arbeit mit Docker und seinen Containern sicher ist. Wie in der \hyperref[dockerIntro]{Einführung zu Docker} beschrieben, stellt Docker zusammen mit anderen Anbietern einen Workflow und eine Palette an Tools zur Verfügung, die die Arbeit mit Containern erleichtern sollen. Wie diese Tools zur Sicherheit bzw. Angreifbarkeit von Docker-Systemen beitragen, wird im Kontext von den Sicherheitszielen betrachtet.
  % Technische und administrative Sicherheit

  %Nicht betrachtet werden die Sicherheitsrisiken, die sich durch den Betrieb eines Rechnernetzwerks ergeben, in dem Docker-Knoten existieren. Sicherheit aus Sicht der Netzwerktechnik und den verschiedenen \acrshort{OSI}-Schichten ist nicht Gegenstand der Untersuchung.

  % TODO: einarbeiten: Siehe \cite[S.3]{virtVSContainer} ... rechts oben ..

  % Da Images aud dem Docker Hub malicious sein koennen ,,,, arbitray code ausgefuehrt wird ...
  % kann auch ein Container, der diesen Code ausfuehrt, nicht vertraut werden /--> kann malicious werden (boesartig, boese absichten verfolgen)

\end{document}
