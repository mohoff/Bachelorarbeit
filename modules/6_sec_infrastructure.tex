\documentclass[../main.tex]{subfiles}
\begin{document}

\chapter{Docker in Unternehmen/Clound-Infrastrukturen}
\label{secInfrastructure}
  Docker wird von den meisten Cloud-Anbietern unterstützt.

  Google: Google Container Engine mit Docker Containern. Mit Kubernetes-Schicht dabei. --> komplette FOSS infrastruktur bei google
  Amazon: Elastic Bean Stack (fuer web applications) und auch EC2-Container als EC2-Container-Service (ECS)

  \section{Microsoft Azure}
    % https://docs.docker.com/v1.9/engine/installation/azure/

    % Ansatz ist generell unschoen, da Azure mit Windows Servern betreiben wird und Docker nur auf Linux laeuft.
    % Deswegen ist jede Azure-Docker-Sache eine Docker-Installation in einer Linux-VM (ubuntu), die auf windows server laeuft

    You can use the portal to add the Docker VM Extension to any compatible Linux VM (currently, the only image that supports it is the Ubuntu 14.04 LTS image more recent than July). Using the Azure CLI command line, however, you can install the Docker VM Extension and create and upload your Docker communication certificates all at the same time when you create the VM instance.
    % Also angeblich nur Ubuntu 14.04 LTS und neuer unterstützt
    \cite{https://azure.microsoft.com/de-de/documentation/articles/virtual-machines-docker-vm-extension/}


    Azure bietet einige VM-Extensions an, darunter Docker Extension, die Docker-Daemon in einer VM installiert.
    Darunter auch viele Extensions für die Sicherheit. Diese umfassen Datenverschlüsselung, Intrusion Detection, Anti-Virus und -Malware, Firewall, Log-Inspection
    \cite{https://azure.microsoft.com/en-us/documentation/articles/virtual-machines-extensions-features/}

    Azure macht Docker auf Basis von Ubuntu-VMs.
    \cite{http://www.infoworld.com/article/2887579/hybrid-cloud/ibm-embraces-docker-openstack-in-bluemix-hybrid-cloud-plans.html}

    Supported Distributions:
      CoreOS
      Ubuntu 13 and higher
      CentOS 7.1 and higher
      Red Hat Enterprise Linux (RHEL) 7.1 and higher
    \cite{https://github.com/Azure/azure-docker-extension}

    % Nonetheless, the competition lacks a genuine hybrid cloud methodology that involves containers, barring whatever someone might create on their own with OpenStack (vendor support not necessarily included). Microsoft is the only one that comes close, with a hybrid strategy that encompasses tight pre-existing integration between Windows locally and Microsoft Azure remotely, though it's limited by the lack of native support for Docker on Windows. Once Microsoft works out how container technologies can be implemented natively in Windows, the picture is bound to change.
    % \cite{http://www.infoworld.com/article/2887579/hybrid-cloud/ibm-embraces-docker-openstack-in-bluemix-hybrid-cloud-plans.html} ... ganz unten


    % https://azure.microsoft.com/en-us/documentation/articles/virtual-machines-docker-with-xplat-cli/


  \section{IBM SoftLayer}
    % https://www.docker.com/IBM
    % https://docs.docker.com/v1.9/engine/installation/softlayer/
    % Treiber: https://docs.docker.com/machine/drivers/soft-layer/
    % http://blog.softlayer.com/tag/docker

    % Wird von moovel für car2Go Carsharing Service verwendet (siehe startseite softlayer)
    % manches auf Basis von OpenStack (http://www.ibm.com/cloud-computing/de/de/infrastructure/)

    % DockerHub ist von SoftLayer gehostet
    % u.a. \cite{http://www.businesscloudnews.com/2014/06/11/ibm-links-up-with-docker-takes-it-to-softlayer/}
    % aber da gibts bestimmt bessere Quellen

    % While many people share images on the public Docker registry, security-minded organizations will want to create a private registry by leveraging SoftLayer object storage. You can create Docker images for a private registry that will store all its information with object storage. Registries are then easy to create and move to new hosts or between data centers.
    \cite{http://blog.softlayer.com/2015/docker-containerization-software}

    With this in mind, Docker partnered with IBM to launch their first commercial offering – Docker Trusted Registry with commercially supported Docker Engines – in June of this year with availability in North America only. This offering is now available in Europe for businesses looking for a commercially supported Docker solution.

    What makes it production ready? In addition to formal support from IBM and Docker, Docker Trusted Registry provides clients with a private registry behind the firewall, security and compliance capabilities to manage access control, and integration with existing directories such as LDAP and Active Directory. It also comes with a user interface for administrators to monitor the health of the registry.

    \cite{https://developer.ibm.com/bluemix/2015/10/23/ibm-and-docker-bring-production-ready-containers-to-europe/}


    % (http://www-03.ibm.com/press/us/en/pressrelease/47165.wss)
    % (http://www-01.ibm.com/common/ssi/ShowDoc.wss?docURL=/common/ssi/rep_ca/1/877/ENUSZP15-0561/index.html&lang=en&request_locale=en)
    % --> SoftLayer Bezug?

    Docker wird in Bluemix eingesetzt und wird unter dem Namen \emph{IBM Containers} vermarktet, also Bluemix-Service.
    % https://developer.ibm.com/bluemix/2015/09/30/ibm-containers-launch-london/
    % https://developer.ibm.com/bluemix/2015/06/22/ibm-containers-on-bluemix/
    % Setpember 2015

    % Delivered as part of Bluemix, IBM’s open cloud platform for application development, the IBM Containers service will enable enterprises to launch Docker containers directly onto the IBM Cloud on bare metal servers from SoftLayer, an IBM company. By leveraging Docker container technology, this will provide companies an environment that is simpler to manage and offers increased utilization and performance in a more flexible execution model, expanding the types of applications that can be supported on the IBM Cloud.
    % \cite{https://www-03.ibm.com/press/us/en/pressrelease/45597.wss}

    SoftLayer vermarketet Multi Tier Security
    \cite{http://www.softlayer.com/security}
    mehrere aus: Intrustion Detection System (IDS), Intrusion Prevention System (IPS), beides kombiniert (IDPS)
    mehrere aus: Firewalls
    Anti-Virus
    Security-Benchmark
    Two-Factor-Authentication
    \cite{http://www.softlayer.com/SECURITY-SOFTWARE}

    Schutz vor DOS, wird von einem NOC-Team aufgezeichnet und "migriert" ... was auch immer das heissen soll, s.u.
    -- A SoftLayer Network Operations Center (NOC) team monitors network performance and security 24x7. Automated DDoS mitigation controls are in place should a DDoS attack occur.
    -- SoftLayer can’t stop a customer from being attacked, but it can shield the customer (and any other customers in the same network) from the effects of the attack. If necessary, SoftLayer will remove the target from the public network for periods of time and null-routes incoming connections. Because of SoftLayer’s three-tiered network architecture, a customer would still have access to the targeted system via the private network.
    -- audit and tracking seitens softlayer fuer alle kunden
    \cite{http://blog.softlayer.com/2014/softlayer-security-questions-and-answers}

    -- Delivered as part of Bluemix, IBM’s open cloud platform for application development, the IBM Containers service will enable enterprises to launch Docker containers directly onto the IBM Cloud on bare metal servers from SoftLayer.

    -- mehr flexibilitaet bei softlayer als bei azure, da sowohl bare metal (gut fuer  such as databases and calculation-intensive applications) als auch virtual server angebote.
    -- "without the overhead of a hypervisor." (http://www.softlayer.com/press/softlayer%E2%84%A2-introduces-bare-metal-cloud%E2%84%A2)
    \cite{http://www.softlayer.com/BARE-METAL-SERVERS ... und ein tab weiter}

  % -------

  % TODO: Docker in/mit VMs

  % Wichtiges Kapitel für Daimler, mein Chef, Management
  % Kapitel, das erst angegangen wird, wenn min. Kapitel 1 steht (Januar 2016 oder später).

  % ???: Transscript aus Besprechung mit Herr Fahner und Patrick:
  % Welche Security-Features uebernimmt die Cloud, welche muss Docker gewaehrleisten. Was
  % bieten MS Azure/Amazon's AWS/etc fuer Mechanismen an?
  % Welche Möglichkeiten zur sicheren Docker-Integration bieten diese?
  % Wortlaut Patrick: Wie funktionierts bei Azure, wie funktionierts wenn man es
  % selbst implementiert.

  % --> mir ist nich klar, was ich da untersuchen kann.

  % Hypervisor-basierte Clouds:
  % Amazon Web Services (AWS) nutzt XEN.
  % Terremark, Savvis, Bluelock nutzen ESXi.
  % AT&T, HP, Comcast, Orange nutzen KVM.
  % Microsoft Azure und MS Private Cloud nutzen hauseigenen Hyper-V.
  % Container-basierte CLouds:
  % Google, IBM/Softlayer, Joyent (RECHERCHIEN: welche Containerplatform sie nutzen. Google glaube ich lmctfy)
  % ^  \cite[S.2]{dockerLXCKub}

  % formale Definitionen von "Cloud Computing" ...
  % "bare metal cloud": Zuweisung von rein physikalischen Servern und Setups. Kein Overhead von Hypervisorn oder Containerization.
  % ^  \cite[S.1]{dockerLXCKub}

  % Containeranwedungen (Service Discovery Tools) wie HAProxy, Zookeeper, etcd und Consul können eingehende Verbindungen auf mehrere Webserver-Container verteilen --> flexibel und skalierbar. Komplizierte Anwendungen sind deswegen realisierbar.
  % ^   \cite[S.4]{dockerIntroIEEE}

  % Tools wie Docker Swarm und Kubernetes erlauben es, Docker-basierte Appliaktionsstacks auch auf mehreren physikalischen Hosts zu realisieren.
  % Prototyp \emph{libswarm}, jetzt Docker Swarm.
  % Auch für Skalierung, Autoskalierung und Redundanz sehr gut
  % ^   \cite[S.4]{dockerIntroIEEE}

  % Moegliche Kombinationen von VMs, OS, Container, Anwedungen sind in \cite[S.3]{dockerLXCKub}

  % Evtl. mit in Ausblick mit rein:
  % However, using containers for security isolation might not be a good idea. In an August 2013 blog, 4 one of Docker’s engineers expressed optimism that containers would eventually catch up to VMs from a security standpoint. But in a presentation given in January 2014, 5 the same engineer said that the only way to have real isolation with Docker is to either run one Docker per host, or one Docker per VM. If high security is needed, it might be worth sacrificing the performance of a pure-container deployment by introducing a VM to obtain more tried and true isolation. As with any other technology, you need to know the deployment’s security requirements, and make appropriate decisions.
  % ^   \cite[S.3]{dockerLXCKub}

  % Container + VM: wird "Defense in depth" genannt
  % ^   \cite[S.14]{presContainerDockerSec}
  % ^   \cite[S.33]{presContainerDockerSec} --> Details

  % Geringes Organisationsrisiko, da zuverlässige Rollbacks immer gemacht werden können durch Dockerfiles, COW
  % ^   \cite[S.35]{presContainerDockerSec}

  % Deploying Docker on your infrastructure:
  % ^   \cite[S.6+7]{dockerSecIntro}

  % Ganz gutes Bild des Aufbaus in virtVSContainer_2014.... s.2 .
  % Auch text in linker spalte

  % Gutes Intro von Docker mit public cloud orechestration:
  % ^   \cite[S.4+5]{virtVSContainer}

  % Die Frage muss nicht sein: VMs oder Container. Man kann auch VM UND Container zusammen betreiben
  % ^   \cite[S.33]{presContainerDockerSec}

  % Datencontainer begünstigen auch das MVC-Pattern, welches nach dem Prinzip \emph{Seperation Of Concerns} die verschiedenen Komponenten
  % Multi-Tier-Architektur
  % Seperation of Concerns...
  % bisschen MVC ...
  % strukturierte Architekturen immer gut, begünstigen Sicherheit und Wartbarkeit ....
  % Verfügbarkeit freut sich auch....

\end{document}
