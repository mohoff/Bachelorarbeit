\documentclass[../main.tex]{subfiles}
\begin{document}

% Kapitel Security Benchmarks?
%   \cite[S.36]{presContainerDockerSec}

% Kapitel Security Policies/Guidelines
%   \cite[S.37]{presContainerDockerSec}

% Aufbauen a la (Prozesse auflisten, un einzeln security features davon erklaeren):
%   1. ISO Download
%   2. Verify check sums
%   3. Install OS
%   4. Prepare OS for imaging
%   5. Create Docker image
%   6. Upload to internal registry server
%   7. Share with others

\chapter{Security im Docker-Ökosystem}
\label{secEcosystem}
  In diesem Kapitel werden einige Anwendungsaspekte von Docker unter einem Sicherheitskontext beleuchtet. Maßgeblich sollen weiterhin die drei definierten Sicherheitsziele zur Bewertung von Sicherheitseigenschaften dienen. Während Kapitel \ref{secLinux} native Sicherheitskomponenten von Docker und Linux untersucht hat, wird in den folgenden Abschnitten das Docker-Ökosystem in Betracht gezogen. Unter einem Ökosystem versteht sich hierbei die Gesamtheit aller Komponenten und Interaktionsmöglichkeiten, die im Zusammenhang mit Docker existieren. Der Fokus der Untersuchung liegt auf Anwendungsebene. Das bedeutet, dass hauptsächlich Methoden vorgestellt werden, die Docker Entwicklern und Administratoren zur Verfügung stellt, um die Arbeit mit den Docker-Komponenten aus Kapitel \ref{dockerIntro} sicher zu gestalten. Außerdem wird vorgestellt, wie sich die sicherheitsrelevanten Komponenten und Operationen in den letzten Monaten aus der Sicherheitsperspektive geändert haben.

  Die Untersuchung sieht folgende Themen vor:

  \begin{itemize}
    \item Verbindung zwischen Docker-Client und Docker-Daemon
    \item Verwaltung von Images
    \item Betrieb von Containern
    \item Verwendung von Plugins
    \item Verwendung von 3rd-Party-Tools, wie z.B. Kubernetes
  \end{itemize}


  % Je nach Vorankommen, können hier ganze Sektions weggelassen werden imo.
  % Tendenziell mehr die Themen zuerst, die direkt mit Security zu tun haben.
  % Auch Fokus auf die neusten Entwicklungen (2015 und 2016) in Sachen Sicherheit und neue Docker-Features

  % High level goals of docker project to improve security
  % • Map root user of the container to non-root user of docker
  % • Make docker daemon run as a non root user
  % ^   \cite[S.3]{virtVSContainer}

  % Support for TUF Delegations: Docker now has support for read/write TUF delegations, and as soon as notary 0.2 comes out, you will be able to use delegations to provide signing capabilities to a team of developers with no shared keys.
  %   ^   \cite{https://news.ycombinator.com/item?id=11037543}

  \section{Image Verifikation}
    Die Verifikation von Daten hat zum Ziel die Integrität dieser zu bestätigen. Die Integrität von Images spielt beim Herunterladen von Images von entfernten Registries eine wichtige Rolle.

    Der Sicherheitsforscher Jonathan Rudenberg hat im Dezember 2014 drei Sicherheitsrisiken im Zusammenhang mit Dockers Image-Verifikation zu dieser Zeit aufgedeckt \cite{https://github.com/docker/docker/issues/9719}\cite{https://titanous.com/posts/docker-insecurity}. U.a. ist es durch Verwendung des \texttt{docker pull}-Befehls möglich manipulierte Images zu beziehen, die bereits beim Entpacken auf dem lokalen System beliebige Dateien im Hostsystem überschreiben können \cite{ https://securityblog.redhat.com/2014/12/18/before-you-initiate-a-docker-pull/ }. Sowohl die Integrität von Daten als auch die Verfügbarkeit der Hosts sind dadurch direkt gefährdet.

    Im August 2015 wurde mit der Docker-Version 1.8 ein neues Paket- und Verteilungsmodell umgesetzt, das diese Schwächen behebt \cite{ https://blog.docker.com/2015/08/content-trust-docker-1-8/ }. Unter dem Featurenamen \emph{Docker Content Trust} integriert Docker das Model \emph{The Update Framework} (TUF) \cite{ https://theupdateframework.github.io/ }, welches Gefahrenquellen wie manipulierte Images, Wiederholungs- und MITM-Angriffe ausschließt. Die Sicherheit von TUF basiert auf der Signierung von Images, mit der anhand mehrerer Schlüsselpaare die Integrität sowie Aktualität von Images sichergestellt wird.

    \emph{Docker Content Trust} wird in Docker als sogenannter Notary integriert.

    https://lwn.net/Articles/628343/
    https://github.com/docker/docker/issues/9719
    https://securityblog.redhat.com/2014/12/18/before-you-initiate-a-docker-pull/
    https://titanous.com/posts/docker-insecurity

    https://github.com/docker/notary
    https://blog.docker.com/2015/08/content-trust-docker-1-8/
    https://blog.docker.com/2015/08/docker-1-8-content-trust-toolbox-registry-orchestration/

  \section{Docker Plugins}
    % Seit Version 1.10
    % Authorization Plugins: you can now write plugins for allowing or denying API requests to the daemon. For example, you could block anyone from using --privileged.
    %   ^   \cite{https://news.ycombinator.com/item?id=11037543} und http://blog.docker.com/2016/02/docker-engine-1-10-security/

  \section{Security Policies und Open Source}
    % Siehe \cite[S.5]{dockerSecIntro} rechts oben

    % Open-Source Charakter von Docker und Security-Disclose...
    %   ^   \cite[S.5]{dockerSecIntro}

    % https://benchmarks.cisecurity.org/tools2/docker/CIS_Docker_1.6_Benchmark_v1.0.0.pdf

    % https://github.com/docker/docker-bench-security

  \section{Lifecycle- und State-Management von Containern}
    % Container Lifecycle - Managed using a daemon and command line tool
    % • Container State - Docker allows ability to store and retrieve container state
    % Siehe \cite[S.3]{virtVSContainer} .. rechte Spalte

  \section{Docker Images und Registries}
    % \cite{slideshareDockercon15}, slide 51
    % \cite[S.27]{presContainerDockerSec} --> Registry v1 und v2, Notary
    % content trust (docker.com/engine/security/trust/content_trust/)

    % Siehe \cite[S.5]{dockerSecIntro}, links

    % Content Addressed Images: The new manifest format in Docker 1.10 is a full Merkle DAG, and all the downloaded content is finally content addressable.
    % Seit Version 1.10
    %   ^   \cite{https://news.ycombinator.com/item?id=11037543}

    % Docker is agnostic to which remote registry is in use, allowing companies to run any combination of homegrown registries, public cloud-based registries or Docker Hub.
    %   ^   \cite[S.5]{dockerSecIntro}

    % Secure communications is also vital to building and shipping applications, as container images are in constant change and need to be pushed and pulled through your infrastructure. All communications with the registries use TLS, to ensure both confidentiality and content integrity. By default, the use of certificates trusted by the public PKI infrastructure is mandatory, but Docker allows the addition of a company internal CA root certificate to the truststore.
    %   ^   \cite[S.5]{dockerSecIntro}

    % local registry mirror
    %   ^   \cite{https://docs.docker.com/v1.6/articles/registry_mirror/}


    \subsection{neues Signierungs-Feature}
	\section{Docker Daemon}
		\subsection{\acrshort{REST}-\acrshort{API}}
		\subsection{Support von Zertifikaten}
  \section{Containerprozesse}
    % TODO: besser in PID namespace kapitel aufgehoben
    % offenbar gibts Probleme mit PID=0, da dieser Besonderheiten aufweist (siehe Hackernews Link)
	\section{Docker Cache}
	\section{\texttt{privileged} Container}
    % However, if the operator executes a container as "privileged", Docker grants access to all devices to the container
    %  ^  \cite[S.4]{dockerSec1}


	\section{Networking}
		\subsection{\texttt{bridge} Netzwerk}
		\subsection{\texttt{overlay} Netzwerk}
		\subsection{DNS}
		\subsection{Portmapping}
	\section{Daten-Container}
	\section{Docker mit \acrshort{VM}s}
  \section{Sicherheitskontrollen für Docker}
    % Gibt auf Github Skripte, die einige Docker-Parameter/Einstellungen prüfen (Links iwo in den Bookmarks)
	\section{Tools rund um Docker}
    % Oder Aufteilung in DevOops Tools (Puppet, Ansible, Vagrant) und Orchestrierungstools (Mesos, Shipyard, Kubernetes)
    % Diese Tools bieten eine weitere Abstraktionsschicht für den Betrieb von Docker-Containern.

    \subsection{Docker-Erweiterungen}
      \subsubsection{Docker Swarm}
      \subsubsection{Docker Compose}
      \subsubsection{Nautilus Project}
    \subsection{Third-Party Tools}
  		\subsection{Vagrant}
  		\subsection{Kubernetes}
        % neueres Cluster-Management Tool von Google
        % Hat Relevanz fuer Herr Fahner/Daimler

        Ein Hauptfeature von Containern ist deren flexibler Einsatz in Anwendungsclustern, die eine Multi-Tier-Anwendung / Multi-Tenant-Architektur abbilden.

        Im Juni 2014 hat Google das Open-Source Tool \emph{Kubernetes} angekündigt, das Cluster mit Docker-Containern verwalten soll. Laut Google ist Kubernetes die Entkopplung von Anwendungscontainern von Details des Hosts.
        Soll in Datencentern die Arbeit mit Containern vereinfachen.
        % Brint angeblich tolles Networking-Feature mit.

        Neben einigen Startups, haben sich \emph{Google}, \emph{Microsoft}, \emph{VMware}, \emph{IBM} und \emph{Red Hat} als \emph{Kubernetes}-Unterstützer geäußert.

\end{document}
