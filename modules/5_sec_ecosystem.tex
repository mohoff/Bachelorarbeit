\documentclass[../main.tex]{subfiles}
\begin{document}

\chapter{Security im Docker-Ökosystem}
\label{secEcosysstem}
  % ### Hier mit Patrick weitermachen ###
  % Je nach Vorankommen, können hier ganze Sektions weggelassen werden imo.
  % Tendenziell mehr die Themen zuerst, die direkt mit Security zu tun haben.
  % Auch Fokus auf die neusten Entwicklungen (2015 und 2016) in Sachen Sicherheit und neue Docker-Features
  \section{Docker Images und Registries}
    % Siehe /cite{slideshareDockercon15}, slide 51
		\subsection{neues Signierungs-Feature}
	\section{Docker Daemon}
		\subsection{REST-API}
		\subsection{Support von Zertifikaten}
  \section{Containerprozesse}
    % offenbar gibts Probleme mit PID=0, da dieser Besonderheiten aufweist (siehe Hackernews Link)
	\section{Docker Cache}
	\section{\texttt{privileged} Container}
	\section{Networking}
		\subsection{\texttt{bridge} Netzwerk}
		\subsection{\texttt{overlay} Netzwerk}
		\subsection{DNS}
		\subsection{Portmapping}
	\section{Daten-Container}
	\section{Docker mit VMs}
  \section{Sicherheitskontrollen für Docker}
    % Gibt auf Github Skripte, die einige Docker-Parameter/Einstellungen prüfen (Links iwo in den Bookmarks)
	\section{Tools rund um Docker}
    % Oder Aufteilung in DevOops Tools (Puppet, Ansible, Vagrant) und Orchestrierungstools (Mesos, Shipyard, Kubernetes)
    % Diese Tools bieten eine weitere Abstraktionsschicht für den Betrieb von Docker-Containern.

    \subsection{Docker-Erweiterungen}
      \subsubsection{Docker Swarm}
      \subsubsection{Docker Compose}
      \subsubsection{Nautilus Project}
    \subsection{Third-Party Tools}
  		\subsection{Vagrant}
  		\subsection{Kubernetes}
        % neueres Cluster-Management Tool von Google
        % Hat Relevanz fuer Herr Fahner/Daimler

        Ein Hauptfeature von Containern ist deren flexibler Einsatz in Anwendungsclustern, die eine Multi-Tier-Anwendung / Multi-Tenant-Architektur abbilden.
        Im Juni 2014 hat Google das Open-Source Tool \emph{Kubernetes} angekündigt, das Cluster mit Docker-Containern verwalten soll. Laut Google ist Kubernetes die Entkopplung von Anwendungscontainern von Details des Hosts.
        Soll in Datencentern die Arbeit mit Containern vereinfachen.
        % Brint angeblich tolles Networking-Feature mit.

        Neben einigen Startups, haben sich Google, Microsoft, VMware, IBM und Red Hat als \emph{Kubernetes}-Unterstützer geäußert.

\end{document}
