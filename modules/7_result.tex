\documentclass[../main.tex]{subfiles}
\begin{document}

\chapter{Fazit}
\label{result}
  % Wenig Angriffsvektoren auf Docker/Container bekannt. "Sichergehen" kann man nur mit Konzepten wie "Docker in VMs" und "VMs in Docker".
  % neuste Docker-Releases und deren Fokus (Enterprise,Production-Readiness,Security)
  % seit Juli 2015 ist standardisiertes Containerformat der big player in Arbeit
  % Ausblick im Kontext von Container VS. konventioneller Virtualisierung

  Spekulation in der Industrie ist, dass sich Organisationen und Unternehmen zusammenschließen und sich auf eine neue, universale Lösung einigen, die die heutigen Fähigkeiten der sich ergänzenden Technologien Docker und Kubernetes, abdeckt \cite[S.4]{dockerLXCKub}.

  % Gutes Intro von Docker mit public cloud orechestration
  % Und: "next gen" "paas" gelabber
  % ^   \cite[S.4+5]{virtVSContainer}
\end{document}
