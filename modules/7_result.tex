\documentclass[../main.tex]{subfiles}
\begin{document}

\chapter{Fazit}
\label{result}
  % Wenig Angriffsvektoren auf Docker/Container bekannt. "Sichergehen" kann man nur mit Konzepten wie "Docker in VMs" und "VMs in Docker".
  % neuste Docker-Releases und deren Fokus (Enterprise,Production-Readiness,Security)
  % seit Juli 2015 ist standardisiertes Containerformat der big player in Arbeit
  % Ausblick im Kontext von Container VS. konventioneller Virtualisierung

  Spekulation in der Industrie ist, dass sich Organisationen und Unternehmen zusammenschließen und sich auf eine neue, universale Lösung einigen, die die heutigen Fähigkeiten der sich ergänzenden Technologien Docker und Kubernetes, abdeckt \cite[S.4]{dockerLXCKub}.

  % Gutes Intro von Docker mit public cloud orechestration
  % Und: "next gen" "paas" gelabber
  % ^   \cite[S.4+5]{virtVSContainer}

  % Docker kauft Unikernel Systems
  % Siehe bookmarks auf techcrunch. Auch gute Grafik dazu..
  % Deutung: Docker weiss, dass Container allein unsicher sind, un mit Unikernels Docker-Container direkt auf Hypervisorn betreiben will
  % --> Mix von Hypervisor und Containern. Damit ist Kernel nicht mehr shared, sondern jeder Container hat sein eigenen minimal kernel.
  % Siehe bookmarks und blog.docker january 2016

  % "Security"-Battle zwischen Docker (viele security features am 04.02.201 mit version 1.10) und dem release von CoreOS 1.0 am selben tag (?), das sich als
  % sichere Containerloesung mit rkt vermarktet.
  %   ^   \cite{alles mit urls belegbar...} --> schon geschehen in kapitel LSM/MAC

  % generelles security profile generator tool in diskussion
  % Um sehr technische details von selinux,apparmor,seccomp,caps zu abstrahieren. Es soll schnittstelle entstehen, mit denen jeder entwickler moeglichst einfach eigenen containersicherheitsprofil festlegt
  % das apparmor oder seccomp dynamisch anwendet, je nach support vom host.
  %   ^   \cite{frazelli/bane tool .... v.1.10 blog post ..... https://github.com/docker/docker/issues/17142#issuecomment-148974642}
	% TODO: Eigene AppArmor profile erstellen mit 'bane': https://github.com/jfrazelle/bane ... Tooling-Efforts von Docker in Sachen Security support
	% Auch: proposal als neues profil generator sec blabla: http://blog.docker.com/2016/02/docker-engine-1-10-security/

  % Kompatibilitaet zwischen den vielen MACS/LSMs/Seccomp/Caps/etc. ? Diese in Zukunft sichergestellt?

  % LSM bald ersetzt durch Seccomp-style oder vereinheitliches, im kernel standardmaessig integriertes security-framework???
  %   ^   \cite{https://lwn.net/Articles/443099/}

  % EIGENE GEDANKEN:
  % dockers monetarisierung schwaechelt, da shift zu einem level drueber --> kubernetes. Docker swarm/compose nicht sehr erfolgreich..
  % Und container mit dem OCP/OCI/OCF standardisiert werden.
  % Einigster core value von docker sind paar tools und die registries/docker hub. Docker als "next maven" fuer container images?

  % TODO: im github repo von runc, dem neuen containerstandard werden technologien (namesapces,cgroups,selinux,apparmor) abstrahiert. es ist von "zusätzlichen isolatoren" die rede (siehe SPEC.md oder so in runC), damit ist selinux,seccomp,apparmor,caps gemeint
  %       Außerdem mehr die Rede von "pods", nicht "containern". Ein pod hat ein oder mherere container enthalten. --> Shift von Container zu Pods. Docker wird unwichtiger, Kubernetes wird wichtiger.

\end{document}
