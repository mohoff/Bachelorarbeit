\documentclass[../main.tex]{subfiles}
\begin{document}

\chapter{Fazit}
\label{result}
  % Wenig Angriffsvektoren auf Docker/Container bekannt. "Sichergehen" kann man nur mit Konzepten wie "Docker in VMs" und "VMs in Docker".
  % neuste Docker-Releases und deren Fokus (Enterprise,Production-Readiness,Security)
  % seit Juli 2015 ist standardisiertes Containerformat der big player in Arbeit
  % Ausblick im Kontext von Container VS. konventioneller Virtualisierung

  Wie die Vielzahl an vorgestellten Sicherheitsmechanismen aus Kapitel \ref{secLinux} zeigt, setzt Docker eine komplexe, mehrschichtige Sicherheitsarchitektur nach den Prinzipien \emph{Defense In Depth} und \emph{Principle Of Least Privilege} auf Softwarebasis um. Die verschiedenen Mechanismen werden wahrscheinlich in einem zukünftigen Docker-Release im Rahmen eines benutzerfreundlichen Werkzeugs zusammengefasst. Dieses soll auf der Basis eines universellen Sicherheitsprofils, die einzelnen Sicherheitstechnologien entsprechend konfigurieren \cite{githubGeneralSecProfiles}. Auch das sicherheitsrelevante Plugin-Framework soll in Docker-Version 1.11 überarbeitet werden \cite{githubDockerRoadmap}\cite{githubAuthZPluginInfrastructure}.

  Diese, in Abschnitt \ref{questionRealization} als technische Kontrollen definierten Maßnahmen, werden durch weitere technische und administrative Konzepte aus Kapitel \ref{secEcosystem} ergänzt. Eine Geheimhaltung von technischen Sicherheitsmaßnahmen kann, wie in Kapitel \ref{opensource} erörtert, nicht praktiziert werden.

  Wie die Ergebnisse aus Kapitel \ref{secInfrastructure} zeigen, bieten alle bekannten Anbieter von Public Clouds sowie \emph{OpenStack} als Lösungsansatz von Private Clouds, dokumentierte Integrationsmöglichkeiten für Docker an. Spezielle Sicherheitsmerkmale in Cloud"=Infrastrukturen existieren innerhalb eines Docker"=Hostsystems nicht. Allgemeine, sicherheitsrelevante Merkmale innerhalb dieses Systems wurden in Kapitel \ref{secLinux} und \ref{secEcosystem} behandelt. In Cloud"=Infrastrukturen rücken vielmehr die Netzwerksicherheit sowie Verwaltungsaspekte in den Vordergrund, die im Fall einiger Public Cloud-Anbietern mit separaten Komponenten realisierbar sind.
  % Diese Behauptung ist nur dann gültig, wenn eine in Kapitel \ref{...} vorgeschlagene Einteilung von Kunden und Anwendungen mit VMs und Containern

  \clearpage

  Ein weiterer wichtiger Punkt in der Sicherheit von Cloud"=Infrastrukturen ist die Art des Cloud-Angebots: Die am meisten verbreiteten Typen SaaS, PaaS und IaaS weisen in ihrer Natur unterschiedliche Sicherheitsmerkmale auf. Da man als Nutzer von IaaS einen hohen Freiheitsgrad bei der Architektur eigener Dienste erhält, sind auch Sicherheitseigenschaften der genutzten Infrastruktur eigenverantwortlich umzusetzen. Diese Verantwortung wird durch den engeren Rahmen von SaaS und PaaS generell auf den Betreiber der Infrastruktur übertragen.

  Docker, welches unter PaaS und IaaS verwendet werden kann, unterscheidet sich nicht unter beiden Angebotstypen. Damit sind auch die Sicherheitseigenschaften von Docker für PaaS und IaaS identisch. Spezielle Anforderungen an die Sicherheit für Anwendungen können umgesetzt werden, indem die in Kapitel \ref{secLinux} und \ref{secEcosystem} vorgestellten Methoden und Mechanismen aktiviert werden oder weitere Sicherheitsinstanzen außerhalb des Docker-Ökosystems hinzugezogen werden. Letztere können z.\,B. Firewalls oder eine Authentifizierung über \emph{Kerberos} darstellen.

  % Aus alleiniger Sicherheitssicht die Frage: Ist Sicherheit essentiell, verzichtet ein Unternehmen lieber auf Performancegewinne bei Cotnainern und setzt auf bewaehrte Hypervisor.
  % Allerdings sehen aktuelle Entwicklungen vor, Container innerhalb von Hypervisor-basierten Gastsystemen zu betreiben, da der durch Container ermöglichte Workflow mit CI und CD durch v.\,a. hohen Automatisierungsgrad sehr effizient ist.

  % Mit Blick auf die Zukunft, sind einige Trends für die Cloud aus aktuellen Veröffentlichungen von Software, Diskussionen, Blog-Artikeln und der Gründung und Aquirierung von Startups, abzulesen.

  Bereits im Docker-Projekt existiert eine firmenübergreifende, interdisziplinäre Zusammenarbeit von Entwicklern, um die Technologie zu erweitern und abzusichern. Durch die Kooperation von Organisationen und Unternehmen der letzten Monate im Rahmen des \acrshort{OCP} weitete sich die Zusammenarbeit aus, um ein standardisiertes und universales Containerformat zu erstellen und zu etablieren.

  \emph{Docker} ist dadurch gezwungen, sein aktuelles Geschäftsmodell zu überdenken. Mit dem quelloffenen OCF wird ein Industriestandard geschaffen, mit dem sich Docker nur noch am Rande profilieren kann. Durch die Standardisierung von Containern ist zu erwarten, dass sich das ursprüngliche Innovationspotential von Containern, das allein \emph{Docker} ausnutzen konnte, auf die Orchestrierung und Verwaltung von Containern verschiebt. \emph{Docker} wird es in diesem Bereich schwieriger haben den Markt zu dominieren, da sich die Konkurrenz zu \emph{Docker Swarm} mit \emph{Kubernetes} von \emph{Google} und \emph{Marathon} von \emph{Mesosphere} bereits heute großer Beliebtheit erfreut.

  \clearpage

  Als Reaktion seitens \emph{Docker} zeigen die jüngsten Aktivitäten im eigenen Blog, dass das Unternehmen den öffentlichen Fokus seiner Arbeit zunehmend auf Unternehmenslösungen und die Bereitstellung von Images und Werkzeugen setzt.

  Die für die Sicherheit interessanten Aktivitäten im Docker-Blog von Oktober 2015 bis März 2016 sowie Beiträge auf der \emph{GitHub}-Plattform im gleichen Zeitraum, bestätigen diesen Trend. Diese öffentlichen Aktivitäten sind in folgender Auflistung stichpunktartig genannt:

  \begin{itemize}
    \item Verbesserung der bestehenden Plattform:
      \begin{itemize}
        \item Erweiterbarkeit von Docker durch Plugins: ein generisches Plugin-Framework wird für Docker-Version 1.11 erwartet \cite{githubDockerRoadmap}\cite{githubAuthZPluginInfrastructure}.
        \item Universelles, einfach zu bedienendes Werkzeug zur Erstellung von Sicherheitsprofilen für \emph{SELinux}, \emph{AppArmor} und \emph{Seccomp} angekündigt \cite{githubGeneralSecProfiles}.
        %\item Native Unterstützung für Windows \cite{dockerWindowsSupport}.
        %\item Integration der OCF-Spezifikationen \cite{https://blog.docker.com/2015/12/containerd-daemon-to-control-runc/}\cite{https://blog.docker.com/2015/12/progress-report-open-container-initiative/}.
        %\item Trennung der Docker-CLIs
        % Aufspaltung der Docker-Komponenten in Docker-Distribution:
        % Aufspaltung in Docker Distribution, das Toolchain enthaelt
        % Fokus auf Security, Reliability, Performance
        % ^   \cite[S.31]{http://www.slideshare.net/Docker/docker-48351569}
      \end{itemize}
    \item Werkzeuge zur Orchestrierung, Verwaltung und Skalierung von Containern. Ak­qui­si­ti­on von \emph{Conductant, Inc.} \cite{dockerAurora}.
    \item Ak­qui­si­ti­on von \emph{Unikernel Systems} und Entwicklung von sogenannten Unikernels \cite{dockerUnikernel}.
  \end{itemize}

  Insbesondere der Kauf von \emph{Unikernel Systems} im Januar 2016 ist aus Sicht der Sicherheit sehr interessant. Die Entwickler von \emph{Unikernel Systems} entwerfen Unikernels, die in Docker-Container langfristig eingebaut werden sollen. Unter der Verwendung von Unikernels wird das klassische Konzept eines geteilten Kernels mit einem Konzept ersetzt, das pro Container einen minimalen Unikernel vorsieht. Neben flexiblen Einsatzmöglichkeiten, wirkt sich Unikernels positiv auf die Sicherheit in Containern aus. Gelingt es \emph{Docker} Unikernels in die eigene Technologie standardmäßig zu integrieren, wäre das ein Meilenstein für die Sicherheit von Containern.

  Nicht nur Docker, sondern auch Linux verändert sich. Wie in den Abschnitten \ref{secIsolierung} und \ref{secCgroups} erwähnt, sind Erweiterungen des Linux-Kernels, in Form von neuen Namespaces und einer Überarbeitung der Control Groups, absehbar. Durch die allgemeine Unzufriedenheit über die aktuelle Implementierung von LSMs ist es möglich, dass diese Architektur zukünftig Änderungen unterzogen wird \cite{seccompLWN}. Durch die Mitarbeit vieler Experten, auch z.\,B. von \emph{Red Hat}, sollten zukünftige Linux-Neuerungen schnell in Docker integrierbar sein.
  % EVtl. den ganzen Abschnitt weglassen?

  Docker mag für viele eine zu junge oder nicht ausreichend getestete Technologie sein, um in Produktion laufende Anwendungen zu betreiben. Es muss allerdings beachtet werden, dass sich bei der Konzeption von Infrastrukturen niemand zwischen Containern und der Virtualisierung mit Hypervisoren entscheiden muss. Vielmehr macht in einem sicherheitskritischen Umfeld eine Kombination beider Virtualisierungstechniken Sinn. Die Sicherheit eines Hypervisors und der von Containern ermöglichte, automatisierbare Workflow lassen sich in einer kombinierten Lösungen gewinnbringend einsetzen. Beide Technologien können gleichzeitig genutzt werden, um neben der Sicherheit auch logische Strukturen in der Infrastruktur abzubilden.


  % Mehr auf Unikernel eingehen. Eingeständnisse für Schwächen in Sachen Sicherheit?
  % Docker kauft Unikernel Systems
  % Siehe bookmarks auf techcrunch. Auch gute Grafik dazu..
  % Deutung: Docker weiss, dass Container allein unsicher sind, un mit Unikernels Docker-Container direkt auf Hypervisorn betreiben will
  % --> Mix von Hypervisor und Containern. Damit ist Kernel nicht mehr shared, sondern jeder Container hat sein eigenen minimal kernel.
  % Siehe bookmarks und blog.docker january 2016

  %Organisationen und Unternehmen schließen sich zusammen, um ein standardisiertes und universales Containerformat zu implementieren.
  %\cite[S.4]{dockerLXCKub}
  % Bessere Quelle: OCF seite, github repo dazu.

  % Gutes Intro von Docker mit public cloud orechestration
  % Und: "next gen" "paas" gelabber
  % ^   \cite[S.4+5]{virtVSContainer}

  % "Security"-Battle zwischen Docker (viele security features am 04.02.201 mit version 1.10) und dem release von CoreOS 1.0 am selben tag (?), das sich als
  % sichere Containerloesung mit rkt vermarktet.
  %   ^   \cite{alles mit urls belegbar...} --> schon geschehen in kapitel LSM/MAC

  % generelles security profile generator tool in diskussion
  % Um sehr technische details von selinux,apparmor,seccomp,caps zu abstrahieren. Es soll schnittstelle entstehen, mit denen jeder entwickler moeglichst einfach eigenen containersicherheitsprofil festlegt
  % das apparmor oder seccomp dynamisch anwendet, je nach support vom host.
  %   ^   \cite{frazelli/bane tool .... v.1.10 blog post ..... https://github.com/docker/docker/issues/17142#issuecomment-148974642}

	% Auch: Bane proposal als neues profil generator sec blabla: http://blog.docker.com/2016/02/docker-engine-1-10-security/

  % (Kompatibilitaet zwischen den vielen MACS/LSMs/Seccomp/Caps/etc. ? Diese in Zukunft sichergestellt?)

  %  LSM bald ersetzt durch Seccomp-style oder vereinheitliches, im kernel standardmaessig integriertes security-framework???
  %   ^   \cite{https://lwn.net/Articles/443099/}

  % EIGENE GEDANKEN:
  % dockers monetarisierung schwaechelt, da shift zu einem level drueber --> kubernetes. Docker swarm/compose nicht sehr erfolgreich..
  % Und container mit dem OCP/OCI/OCF standardisiert werden.
  % Einigster core value von docker sind paar tools und die registries/docker hub. Docker als "next maven" fuer container images?

  % im github repo von runc, dem neuen containerstandard werden technologien (namesapces,cgroups,selinux,apparmor) abstrahiert. es ist von "zusätzlichen isolatoren" die rede (s. SPEC.md oder so in runC), damit ist selinux,seccomp,apparmor,caps gemeint
  %       Außerdem mehr die Rede von "pods", nicht "containern". Ein pod hat ein oder mherere container enthalten. --> Shift von Container zu Pods. Docker wird unwichtiger, Kubernetes wird wichtiger.

  % Docker will plugin infrastructure....
  % https://github.com/docker/docker/pull/15365 ... wurde verlinkt von 1.10 release notes (https://github.com/docker/docker/releases/tag/v1.10.0 > security)

  % TODO: Conclusion 5_sec_ecosystem:
  % Anbieter sind in externen Sicherheitsfeatures wie IDS,IPS,Malware,Viren,DOS sehr aehnlich. Docker ist in jeder Cloud dasselbe Docker, sprich Sicherheit ändert sich nicht und ist im aktuellen Stand mit kapitel 4_sec_linux und 5_sec_ecosystem behandelt. Der Grad an Sicherheit, der pro Anwendung eines Kunden gilt, muss berücksichtigt werden. Aus der Konstellation aus SET1 und SET2 fallen durch mangelende Unterstützung mancher Cloudanbieter, Anbieter weg.
  % Zukünftig kann erwartet werden, dass sich Anbieter auch in Sachen OS-usw-Unterstützung ähnlicher werden. Konkurrenz wird dann über andere Eigenschaften ausgestochen werden. Z.B. Preis. --> Cloud wird zum Massenprodukt, und bis nächste Innovation kommt ist Unterscheidung nur über Preis und Performance möglich, da alle ähnliches Featuresset haben. ... Vergleich mit anderen Dingen, z.\,B. HDD-SDD-Laptop-Hersteller

  % In der Wirtschaft ist Vendor-Lock-In ein Thema. Kunden wollen sich nicht auf eine Technologie beschränken und heissen Portabilität und Interoperabilität zwischen mehreren Anbietern willkommen. Anbietern von Clouds sind deswegen interessiert, Open-Source Produkte zu entwickeln und medienlaut als Unterstützer dieses Oopen-Source-Produkts genannt zu werden. Mit Open-Source-Software ist oft gewährleistet, dass sie von vielen Unternehmen unterstützt wird.
  % Ausserdem stellen Unternehmen so sicher, nicht "den Anschluss in der Technik und aktuellen Entwicklungen zu verpassen"
  % Beleg: Erst Docker / Kubernetes... Jetzt OpenStack
  % --> alle 3 sind open-source und werden von kommerziellen Unternehmen unterstützt bis unter den Unternehmen Einigkeit über neuen Open-Source-Standard herrscht, der breit akzeptiert ist. Konkurrenz spielt sich dann auf anderer Ebene ab (Orchestrierung, Tooling, Performance, Preis), wie die Zukunft zeigen wird.

\end{document}
