\documentclass[11pt,a4paper,oneside]{report}
\usepackage[ngerman]{babel}
\usepackage[utf8]{inputenc} % Displays German 'Umlaute' correctly. Also some workaround, see Bibliography Management#BibTeX in wikibooks
\usepackage{subfiles} % In order to generate final file from multiple .tex source files.
% \usepackage{url} or
\usepackage{hyperref}
\usepackage{graphicx}
%\usepackage{titlesec}
\usepackage[parfill]{parskip}
\usepackage{chngcntr} % Change counter (start numeric page numbering on first content page)
\usepackage{glossaries} % Add parameter [toc] to include glossary in index page. Needs to be defined after package 'hyperref' so glossary occurances are linked.
\makeglossaries % Generates (one or multiple) glossaries


% Enables consecutive figure and table numbering, independent of chapter count
\counterwithout{figure}{chapter}
\counterwithout{table}{chapter}

% Maximum depth of shown headlines in index page
\setcounter{tocdepth}{5}
% Maximum depth of numbered headlines
\setcounter{secnumdepth}{3}

% Rewrites name of abstract in German language to "Abstract"
\addto{\captionsngerman}{\renewcommand*{\abstractname}{Abstract}}
%\renewcommand{\abstractname}{Abstract}

% Some global symbols
\newcommand{\bibtexFile}{./modules/bibliography} % written without extension
\newcommand{\bibtexFilePath}{./modules/bibliography.bib} % written without extension
\newcommand{\thema}{Sicherheitsbetrachtungen von Applikations-Containersystemen in Cloud-Infrastukturen am Beispiel Docker}
\newcommand{\fig}{Abb.}
\newcommand*{\signatureAndDate}{
    \par\noindent\makebox[2.5in]{\hrulefill} \hfill\makebox[2.0in]{\hrulefill}%
    \par\noindent\makebox[2.5in][l]{Unterschrift}      \hfill\makebox[2.0in][l]{Datum}%
}

% Some meta data
\title{\thema}
\author{Moritz Hoffmann\\
  Studiengang Mobile Medien,\\
  Hochschule der Medien\\
  \texttt{mh203@hdm-stuttgart.de}}
\date{\today}

\subfile{./modules/glossary}

% Content
\begin{document}
  \pagenumbering{gobble} % No page numbers on the first pages

  % Deckblatt
  \subfile{./modules/cover}

  %\pagenumbering{roman}  % Aktiviert römische Ziffern als Seitenzahlen

  % Eidesstattliche Erklärung
  \subfile{./modules/declaration}

  % Inhaltsverzeichnis (TOC)
  \tableofcontents

  % Abbildungsverzeichnis
  \listoffigures

  % Tabellenverzeichnis
  \listoftables

  % Struktur der Arbeit
  \subfile{./modules/1_structure}
  \pagenumbering{arabic}

  % Einführung
  \subfile{./modules/2_intro}

  % Forschungsfrage
  \subfile{./modules/3_question}

  % Security aus Linux Kernel Features
  \subfile{./modules/4_sec_linux}

  % Security im Docker Ökosystem
  \subfile{./modules/5_sec_ecosystem}

  % Security in Cloud-Infrastrukturen
  \subfile{./modules/6_sec_infrastructure}

  % Fazit
  \subfile{./modules/7_result}

  % Glossar
  \printglossary

  % Anhang (falls benötigt)
  \appendix
    % Evtl. beispielhafte Dockerfiles anzeigen (oder das inline)

  % Quellenverzeichnis
  \bibliographystyle{plain}
  \bibliography{\bibtexFile} % specify name of .bib file. Use of this is deprecated.
  % NOT DEPRACTED, BUT NOT WORKING :/
  %\addbibresource{\bibtexFilePath}
\end{document}
